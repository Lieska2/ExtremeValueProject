\chapter{Convergence in distribution with cdfs}

This chapter provides a standard characterization of convergence in distribution
(weak convergence of probability measures) on the real line in terms of
cumulative distribution functions.

\section{Convergence in distribution}

Convergence in distribution for random variables can be defined when the random variables
take values in a topological space, and it amounts to the weak convergence of the
probability measures that are the laws of those random variables. In the special case
of real-valued random variables, or probability measures on the real line, the definition
reads:
\begin{definition}[Weak convergence of probability measures]
  \label{def:convergence-in-distribution}
  %\uses{}
  \lean{MeasureTheory.ProbabilityMeasure.instTopologicalSpace,
  MeasureTheory.ProbabilityMeasure.tendsto_iff_forall_lintegral_tendsto}
  \leanok
  \mathlibok
  A sequence $(\PRmeas_n)_{n \in \bN}$ of Borel probability measures on $\bR$ converges
  weakly to a Borel probability measure $\PRmeas$ on $\bR$ if for all bounded continuous
  functions $f \colon \bR \to [0,+\infty)$ we have
  \begin{align*}
    \lim_{n \to \infty} \int_{\bR} f(x) \, \ud \PRmeas_n(x)
    = \int_{\bR} f(x) \, \ud \PRmeas(x) .
  \end{align*}
\end{definition}

\section{Auxiliary results}

\begin{lemma}[Monotone real functions have only countably many points of discontinuity]
  \label{lem:monotone-discontinuities-countable}
  %\uses{}
  \lean{Set.Countable.dense_compl, Monotone.countable_not_continuousAt}
  \leanok
  \mathlibok
  A monotone function $f \colon \bR \to \bR$ can have at most countably
  many points of discontinuity. In particular the set $D \subset \bR$
  of continuity points of $f$ is dense in $\bR$.
\end{lemma}
\begin{proof}
  %\uses{}
  % \leanok
  (The proof should already be in Mathlib.)
\end{proof}

\begin{lemma}[Tightness of a cumulative distribution function]
  \label{lem:cdf-tight}
  \uses{def:cdf}
  \lean{CumulativeDistributionFunction.forall_pos_exists_exists_lt_gt_continuousAt}
  % \leanok
  Let $F$ be a cumulative distribution function. Then for any
  $\eps > 0$ there exists points $a,b \in \bR$ with $a < b$ such that
  $F(b) - F(a) > 1 - \eps$ and $F$ is continuous at the points $a$ and $b$.
\end{lemma}
\begin{proof}
  \uses{def:cdf, lem:monotone-discontinuities-countable}
  % \leanok
  Cumulative distribution functions satisfy
  $F(x) \downarrow 0$ as $x \downarrow - \infty$ and
  $F(x) \uparrow 1$ as $x \uparrow + \infty$. The
  required large difference $F(b) - F(a)$ is
  obtained by choosing $a$ small enough so that $F(a) < \frac{\eps}{2}$
  and $b$ large enough so that $F(b) > 1 - \frac{\eps}{2}$.
  In order to guarantee that $a < b$ and
  that $a$ and $b$ are continuity points of $F$, we
  recall that continuity points of the monotone function $F$ are dense
  by Lemma~\ref{lem:monotone-discontinuities-countable}, so we
  may decrease $a$ and increase $b$ as appropriate.
\end{proof}

\begin{lemma}[Subdivision with small mesh and within dense set]
  \label{lem:subdivision-dense}
  %\uses{}
  \lean{forall_exists_subdivision_diff_lt_of_dense}
  % \leanok
  Let $D \subset \bR$ be a dense set and $a,b \in D$ with $a < b$.
  Then for any $\delta > 0$ there exists a $k \in \bN$
  and $a = c_0, c_1, \ldots, c_{k-1}, c_k = b \in D$ such that
  $|c_j - c_{j-1}| < \delta$ for all $j=1,\ldots,k$.
\end{lemma}
\begin{proof}
  %\uses{}
  % \leanok
  \ldots
\end{proof}

\begin{lemma}[Subdivision for continuous function approximation]
  \label{lem:continuous-function-approximation-subdivision}
  %\uses{}
  \lean{forall_exists_subdivision_dist_apply_lt_of_dense_of_continuous}
  % \leanok
  Let $D \subset \bR$ be a dense set, let $f \colon \bR \to \bR$ be continuous,
  let $a, b \in D$ with $a < b$, and let $\eps > 0$.
  Then there exists a $k \in \bN$ and points
  $a=c_0 < c_1 < \cdots < c_{k-1} < c_k = b$ such that for
  each $j = 1, \ldots, k$ we have $c_j \in D$ and
  \begin{align*}
  \big| f(x) - f(c_j) \big| < \eps
  \qquad \text{ for } \qquad x \in [c_{j-1} , c_j] .
  \end{align*}
\end{lemma}
\begin{proof}
  \uses{lem:subdivision-dense}
  % \leanok
  On the compact interval
  $[a,b] \subset \bR$, the continuous function~$f$ is uniformly continuous, so for some
  $\delta > 0 $ we have $|f(x)-f(y)|<\eps$ whenever $|x-y|<\delta$ and $x,y \in
  [a,b]$.
  Now apply Lemma~\ref{lem:subdivision-dense} to choose $k$ and points
  $a=c_0 < c_1 < \cdots < c_{k-1} < c_k$ such that $c_j - c_{j-1} < \delta$
  and $c_j \in D$ for all $j = 1 , \ldots , k$.
  Now for any $j = 1 , \ldots , k$, since for $x \in [c_{j-1} , c_j]$ we have
  $|x - c_j| < \delta$, we get
  \begin{align*}
  \big| f(x) - f(c_j) \big| < \eps
  \end{align*}
  as desired.
\end{proof}

\begin{lemma}[Simple function integral as linear combination of cdf differences]
  \label{lem:simple-integral-cdf-difference}
  \uses{def:cdf}
  \lean{CumulativeDistributionFunction.integral_sum_indicator_eq}
  % \leanok
  Let $a = c_0 < c_1 < \cdots < c_k = b$ and consider the
  linear combination of indicator functions
  \begin{align*}
    h(x) = \sum_{j=1}^k \alpha_j \; \indof{(c_{j-1},c_j]}(x) .
  \end{align*}
  Then the integral of $h$ with respect to a Borel probability measure $\PRmeas$ on $\bR$ whose
  can be written as
  \begin{align*}
    \int_{\bR} h(x) \, \ud \PRmeas (x) = \sum_{j=1}^k \alpha_j \, \big( F(c_j) - F(c_{j-1}) \big) ,
  \end{align*}
  where $F$ is the c.d.f. of $\PRmeas$.
\end{lemma}
\begin{proof}
  % \uses{}
  % \leanok
  \begin{align*}
  \int_{\bR} h \, \ud \PRmeas
   = \; & \int_{\bR}
          \Big( \sum_{j=1}^k \alpha_j \, \indof{(c_{j-1},c_j]}(x) \Big) \, \ud \PRmeas(x) \\
   = \; & \sum_{j=1}^k \alpha_j \; \int_{\bR} \indof{(c_{j-1},c_j]}(x) \, \ud \PRmeas(x) \\
   = \; & \sum_{j=1}^k \alpha_j \; \PRmeas \big[ (c_{j-1},c_j] \big] \\
   = \; & \sum_{j=1}^k \alpha_j \, \big( F_n(c_j) - F_n(c_{j-1}) \big)
  \end{align*}
\end{proof}

\section{Convergence in distribution from pointwise convergence of cdfs}

\begin{theorem}[Characterization of convergence in distribution with cdfs]
  \label{thm:convergence-in-distribution-with-cdf}
  \uses{def:convergence-in-distribution, def:cdf}
  %\lean{}
  % \leanok
  Let $F$ and $F_n$, $n \in \bN$, be cumulative distribution functions of
  probability measures $\PRmeas$ and $\PRmeas_n$, $n \in \bN$, respectively, i.e.,
  \begin{align*}
      F(x) = \; & \PRmeas \big[(-\infty,x]\big] & & \text{for $x \in \bR$} \\
      F_n(x) = \; & \PRmeas_n \big[(-\infty,x]\big] & & \text{for $x \in \bR$ and $n \in \bN$.}
  \end{align*}
  If $\lim_{n \to \infty} F_n(x) = F(x)$ for all continuity points $x$ of $F$,
  then $\lim_{n \to \infty} \PRmeas_n = \PRmeas$ in the sense of weak convergence
  of measures, Definition~\ref{def:convergence-in-distribution}.
\end{theorem}
\begin{proof}
  \uses{lem:monotone-discontinuities-countable, lem:cdf-tight,
  lem:continuous-function-approximation-subdivision,
  lem:simple-integral-cdf-difference}
  % \leanok
  Let $D \subset \bR$ denote the set of continuity points of $F$. By
  Lemma~\ref{lem:monotone-discontinuities-countable}, $D$ is dense in $\bR$.
  Assume that $\lim_{n \to \infty} F_n(x) = F(x)$ for all $x \in D$.

  Let~$\eps > 0$. Choose, by Lemma~\ref{lem:cdf-tight}, points
  $a,b \in D$, $a<b$, such that $F(b) - F(a) > 1 - \eps$.

  Observe also that since $\lim_{n \to \infty} F_n(a) = F(a)$
  and $\lim_{n \to \infty} F_n(b) = F(b)$, there exists some $N_1$ such that
  we have
  \begin{align*}
  F_n(b) - F_n(a) > 1 - 2\eps \qquad \text{ for all } n \geq N_1 .
  \end{align*}

  Let $f \colon \bR \to \bR$ be bounded and continuous.
  By Lemma~\ref{lem:continuous-function-approximation-subdivision}
  we can choose points
  $a=c_0 < c_1 < \cdots < c_{k-1} < c_k = b$ such that
  for all $j = 1 , \ldots , k$ we have $c_j \in D$ and
  \begin{align*}
  \big| f(x) - f(c_j) \big| < \eps
  \qquad \text{ for } \qquad x \in [c_{j-1} , c_j] .
  \end{align*}
  Define the simple function $h \colon \bR \to \bR$ by
  \begin{align*}
  h(x) = \sum_{j=1}^k f(c_j) \; \indof{(c_{j-1},c_j]}(x)
  \end{align*}
  The above estimate shows that $|f(x) - h(x)| < \eps$ for all $x \in [a,b]$.
  By boundedness of~$f$, there exists a constant $K>0$ such that $|f(x)| \leq K$
  for all $x \in \bR$. Since $h$ vanishes outside $(a,b]$,
  the triangle inequality for integral with respect to $\PRmeas_n$ gives
  \begin{align*}
  \Big| \int_{\bR} f \, \ud \PRmeas_n - \int_{\bR} h \, \ud \PRmeas_n \Big|
  \, \leq \; & \, \underbrace{\int_{(a,b]} |f-h| \, \ud \PRmeas_n}_{\leq \eps}
      + \underbrace{\int_{\bR \setminus (a,b]} |f| \, \ud \PRmeas_n}_{%
      \leq K \, \PRmeas_n\big[ \bR \setminus (a,b] \big]} .
  \end{align*}
  When $n \geq N_1$, we have
  $\PRmeas_n\big[ \bR \setminus (a,b] \big]
  = 1 - \PRmeas_n\big[ (a,b] \big] = 1 - (F_n(b) - F_n(a)) < 2 \eps$,
  and thus the triangle inequality implies
  \begin{align*}
  \Big| \int_{\bR} f \, \ud \PRmeas_n - \int_{\bR} h \, \ud \PRmeas_n \Big|
  \, \leq \; & \eps + K \, 2 \eps = (1 + 2K) \, \eps .
  \end{align*}
  Similarly, integrating now with respect to $\PRmeas$ instead, one shows that
  \begin{align*}
  \Big| \int_{\bR} f \, \ud \PRmeas - \int_{\bR} h \, \ud \PRmeas \Big|
  \, \leq \; & (1 + K) \, \eps .
  \end{align*}
  It remains to consider the integrals of the function~$h$ with respect to both
  $\PRmeas_n$ and $\PRmeas$. By Lemma~\ref{lem:simple-integral-cdf-difference},
  these integrals are expressible in terms of the cumulative distribution functions,
  \begin{align*}
  \int_{\bR} h \, \ud \PRmeas_n
  = \; & \sum_{j=1}^k f(c_j) \, \big( F_n(c_j) - F_n(c_{j-1}) \big)
  \end{align*}
  and
  \begin{align*}
  \int_{\bR} h \, \ud \PRmeas
  % = \; & \sum_{j=1}^k f(c_j) \; \PRmeas_n \big[ (c_{j-1},c_j] \big]
  %      \\
  = \; & \sum_{j=1}^k f(c_j) \, \big( F(c_j) - F(c_{j-1}) \big) .
  \end{align*}
  The difference of the integrals of~$h$ with respect to these two can
  therefore be estimated as
  \begin{align*}
  \Big| \int_{\bR} h \, \ud \PRmeas - \int_{\bR} h \, \ud \PRmeas_n \Big|
  \, = \; & \, \Big| \sum_{j=1}^k f(c_j) \,
      \big( F(c_j) - F_n(c_j) - F(c_{j-1}) +  F_n (c_{j-1}) \big) \Big| \\
  \leq \; & \, \sum_{j=1}^k |f(c_j)| \; \Big(
      \big| F(c_j) - F_n(c_j) \big| + \big| F(c_{j-1}) +  F_n (c_{j-1}) \big|
      \Big) \\
  \leq \; & \, 2 k K \max_{j = 0 , \ldots , k} \big| F(c_j) - F_n(c_j) \big| .
  \end{align*}
  By our assumption~\textup{(ii)}, we have
  $\lim_{n \to \infty} F_n(c_j) = F(c_j)$ for each $j = 1 , \ldots, k$, so
  there exists $N_2$ such that for $n \geq N_2$ we have
  $\max_{j = 1 , \ldots , k} | F(c_j) - F_n(c_j) | < \frac{\eps}{k}$, and
  thus
  \begin{align*}
  \Big| \int_{\bR} h \, \ud \PRmeas - \int_{\bR} h \, \ud \PRmeas_n \Big|
  \leq \; & \, 2 K \eps .
  \end{align*}
  Combining the estimates we have obtained, for $n \geq \max(N_1 , N_2)$, we have
  \begin{align*}
  & \Big| \int_{\bR} f \, \ud \PRmeas - \int_{\bR} f \, \ud \PRmeas_n \Big| \\
  \leq \; \, &
      \underbrace{
      \Big| \int_{\bR} f \, \ud \PRmeas - \int_{\bR} h \, \ud \PRmeas \Big|}_{
          \leq (1+K) \eps}
      + \underbrace{
      \Big| \int_{\bR} h \, \ud \PRmeas - \int_{\bR} h \, \ud \PRmeas_n \Big|}_{
          \leq 2 K \eps}
      + \underbrace{
      \Big| \int_{\bR} h \, \ud \PRmeas_n - \int_{\bR} f \, \ud \PRmeas_n	\Big|}_{
          \leq (1+2K) \eps } \\
  \leq \; \, & (2 + 5 K) \eps .
  \end{align*}
  Since $\eps > 0$ was arbitrary, this shows that
  $\int f \, \ud \PRmeas_n \to \int f \, \ud \PRmeas$ as $n \to \infty$, so we
  have established the weak convergence $\PRmeas_n \to \PRmeas$
  according to Definition~\ref{def:convergence-in-distribution}.
\end{proof}
