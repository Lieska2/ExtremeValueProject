\chapter{Types}

\begin{definition}[Type (of distribution on $\bR$)]
  \label{def:cdf-type}
  \uses{def:cdf, lem:oriented-affine-action-on-cdf}
  %\lean{lcinv}
  %\leanok
  Two c.d.f.s $F, G$ are said to be of the same type, if
  there exists an order-preserving affine isomorphism $A \in \OriAffR$
  such that $G = A . F$.

  Being of the same type is an equivalence relation, and
  the equivalence classes are called types (of distributions on $\bR$).
\end{definition}


\section{Convergence to types}

The notion of convergence of cumulative distribution functions
considered here is always taken to be pointwise convergence
on the set of continuity points of the limit c.d.f.
By Theorem~\ref{thm:convergence-in-distribution-with-cdf} and
Lemma~\ref{lem:cdf-convergence-from-convergence-in-distribution},
this corresponds to convergence in distribution (weak convergence
of probability measures). Below, when we write $F_n \tendsinlaw F$
for c.d.f.s $F_n$, $n \in \bN$, and $F$, this is always what we
mean.

For affine maps $A \colon \bR \to \bR$, we use the topology
of pointwise convergence of functions. Equivalently, convergence of affine maps
means the convergence of the coefficients $a, b \in \bR$ in the expression
$x \mapsto a x + b$ of the affine functions
(so $A_n \to A$ if and only if the functions are of the form
$A_n(x) = a_n x + b_n$ and $A(x) = a x + b$, and $a_n \to a$ and $b_n \to b$).

\begin{lemma}[Unique affine relation among two nondegenerate c.d.f.s]
  \label{lem:unique-affine-relation-cdf}
  \uses{def:cdf, lem:oriented-affine-action-on-cdf}
  % \lean{}
  % \leanok
  Let $F, G$ be two c.d.f.s of the same type, and $A \in \OriAffR$
  an affine isomorphism such that $G = A.F$. If $F$ is nondegenerate,
  then $A$ is the only element of $\OriAffR$ for which the
  relation $G = A.F$ holds.
\end{lemma}
\begin{proof}
  % \uses{}
  % \leanok
  Since $F$ is nondegenerate, we can find two different points $x_1 < x_2$
  such that $0 < F(x_1) < F(x_2) < 1$. By right continuity of $F$, we can
  assume these points to be taken minimal with the given values, i.e.,
  $x_j = \inf \big\{ x \in \bR \; \big| \; F(x) = F(x_j) \big\}$ for
  $j=1,2$.

  The assumption $G = A.F$ means $G(x) = F \big( A^{-1}(x) \big)$ for all $x \in \bR$.
  Therefore $G \big( A(x_1) \big) = F(x_1) < F(x_2) = G \big( A(x_1) \big)$
  and also $A(x_j) = \inf \big\{ y \in \bR \; \big| \; G(y) = F(x_j) \big\}$ for
  $j=1,2$.

  If $\widetilde{A} \in \OriAffR$ is also such that $G = \widetilde{A}.F$, then
  the same holds for it:
  $\widetilde{A}(x_j) = \inf \big\{ y \in \bR \; \big| \; G(y) = F(x_j) \big\}$
  for $j=1,2$. We conclude that
  \begin{align*}
    \widetilde{A}(x_1) \, = \; &
        \inf \big\{ y \in \bR \; \big| \; G(y) = F(x_1) \big\} \; = \, A(x_1) \\
    \widetilde{A}(x_2) \, = \; &
        \inf \big\{ y \in \bR \; \big| \; G(y) = F(x_2) \big\} \; = \, A(x_2) .
  \end{align*}
  But an affine map of $\bR$ is determined by its values at two distinct points:
  from $a x_1 + b = y_1$ and $a x_2 + b = y_2$ with $x_1 \ne x_2$ one can solve $a, b$.
  Therefore we must have $\widetilde{A} = A$.
\end{proof}

\begin{theorem}[Convergence to types]
  \label{thm:convergence-to-types}
  \uses{def:cdf, lem:oriented-affine-action-on-cdf, def:cdf-type}
  % \lean{}
  % \leanok
  Suppose that $(F_n)_{n \in \bN}$  is a sequence of
  c.d.f.s which converges to a nondegenerate c.d.f. $G$, i.e.,
  $F_n \tendsinlaw G$ as $n \to \infty$.
  Let $(A_n)_{n \in \bN}$ be a sequence of oriented
  affine isomorphisms of $\bR$, $A_n \in \OriAffR$
  such that $A_n.F_n \tendsinlaw \widetilde{G}$ for some c.d.f.
  $\widetilde{G}$.

  If we write $A_n(x) = a_n x + b_n$, then $(a_n)_{n \in \bN}$
  and $(b_n)_{n \in \bN}$ are bounded sequences.

  If $\widetilde{G}$ is nondegenerate, then
  $A_n \to A \in \OriAffR$ and $A.\widetilde{G} = G$.
  In particular $G$ and $\widetilde{G}$ are of the same type.
  Moreover, $A$ is the unique affine transformation for which
  the equality $A.\widetilde{G} = G$ holds.
\end{theorem}
\begin{proof}
  \uses{lem:unique-affine-relation-cdf}
  % \leanok
  \ldots
\end{proof}

\begin{theorem}[Convergence to types again]
  \label{thm:convergence-to-types-different-affine}
  \uses{def:cdf, lem:oriented-affine-action-on-cdf, def:cdf-type}
  % \lean{}
  % \leanok
  Let $(A_n)_{n \in \bN}$ and
  $(\widetilde{A}_n)_{n \in \bN}$ be two sequences of oriented
  affine isomorphisms of $\bR$, $A_n, \widetilde{A}_n \in \OriAffR$,
  written as $A_n(x) = a_n x + b_n$
  and $\widetilde{A}_n(x) = \tilde{a}_n x + \tilde{b}_n$.

  Let $(F_n)_{n \in \bN}$ be a sequence of c.d.f.s
  such that $A_n.F_n \tendsinlaw G$, with
  $G$ a nondegenerate c.d.f.

  Then the convergence of also $\widetilde{A}_n.F_n \tendsinlaw G$
  holds if and only if the coefficients of the affine maps
  satisfy the relations
  \begin{align*}
    \frac{\tilde{a}_n}{a_n} \to 1
    \quad \text{ and } \quad
    \frac{\tilde{b}_n - b_n}{a_n} \to 0 .
  \end{align*}
\end{theorem}
\begin{proof}
  \uses{thm:convergence-to-types}
  % \leanok
  \ldots
\end{proof}

\begin{corollary}[Convergence to types with different limits]
  \label{cor:convergence-to-types-different-limit}
  \uses{def:cdf, lem:oriented-affine-action-on-cdf, def:cdf-type}
  % \lean{}
  % \leanok
  Let $(A_n)_{n \in \bN}$ and
  $(\widetilde{A}_n)_{n \in \bN}$ be two sequences of oriented
  affine isomorphisms of $\bR$, $A_n, \widetilde{A}_n \in \OriAffR$,
  written as $A_n(x) = a_n x + b_n$
  and $\widetilde{A}_n(x) = \tilde{a}_n x + \tilde{b}_n$.

  Let $(F_n)_{n \in \bN}$ be a sequence of c.d.f.s
  such that $A_n.F_n \tendsinlaw G$ and
  $\widetilde{A}_n.F_n \tendsinlaw \widetilde{G}$, with
  $G$ and $\widetilde{G}$ nondegenerate c.d.f.s.
  Then for some $\alpha>0$ and $\beta \in \bR$ we have
  \begin{align*}
    \frac{\tilde{a}_n}{a_n} \to \alpha
    \quad \text{ and } \quad
    \frac{\tilde{b}_n - b_n}{a_n} \to \beta ,
  \end{align*}
  and we have
  \begin{align*}
    A.G = \widetilde{G}
    \qquad \text{ where } \; A(x) = \alpha x + \beta .
  \end{align*}
  In particular, $G$ and $\widetilde{G}$ have the same type.
\end{corollary}
\begin{proof}
  \uses{thm:convergence-to-types-different-affine}
  % \leanok
  Suppose that $A_n.F_n \tendsinlaw G$ and
  $\widetilde{A}_n.F_n \tendsinlaw \widetilde{G}$.

  Applying the convergence to types theorem,
  Theorem~\ref{thm:convergence-to-types}, to the sequence $(A_n.F_n)_{n \in \bN}$
  of c.d.f.s and to the sequence of transformations
  $\big( \widetilde{A}_n \circ A_n^{-1} \big)_{n \in \bN}$, we get that
  $G$ and $\widetilde{G}$ must be of the same type, i.e., there exists a
  $A \in \OriAffR$ such that $A.\widetilde{G} = G$.

  We then get
  \begin{align*}
    (A \circ \widetilde{A}_n).F_n = A.(\widetilde{A}_n.F_n) \tendsinlaw A.\widetilde{G} = G
  \end{align*}
  (by continuity of the action of affine transforms on distributions).
  Theorem~\ref{thm:convergence-to-types-different-affine} then applies.
  %To express its conclusion conveniently, write $A^{-1}(x) = \alpha x + \beta$
  %(so that $A(x) = \alpha^{-1} (x - \beta)$)
  %and $\widetilde{A}_n(x) = \tilde{a}_n x + \tilde{b}_n$ and
  %$A_n(x) = a_n x + b_n$. Then we have
  %\begin{align*}
  %  (A \circ \widetilde{A}_n)(x) = \alpha^{-1} \big( \tilde{a}_n x + \tilde{b}_n - \beta \big)
  %    = \alpha^{-1} \tilde{a}_n x + \alpha^{-1} (\tilde{b}_n - \beta) ,
  %\end{align*}
  %and the conclusion of Theorem~\ref{thm:convergence-to-types-different-affine} is
  %\begin{align*}
  %  \frac{\alpha^{-1} \tilde{a}_n}{a_n} \to 1
  %  \quad \text{ and } \quad
  %  \frac{\alpha^{-1} (\tilde{b}_n - \beta) - b_n}{a_n} \to 0 .
  %\end{align*}
  \ldots
\end{proof}

\begin{lemma}[A choice of normalizing constants for convergence to types]
  \label{lemma:convergence-to-types-normalization}
  \uses{def:cdf, lem:oriented-affine-action-on-cdf, def:cdf-type, def:lc-inverse}
  % \lean{}
  % \leanok
  (It is possible to choose normalization constants for the affine
  transformations using the left-continuous inverses of the c.d.f.s.
  TODO: Precise statement.)
%  Suppose that $(F_n)_{n \in \bN}$ is a sequence of c.d.f.s
%  and $(A_n)_{n \in \bN}$ %and $(\widetilde{A}_n)_{n \in \bN}$
%  be a sequence of oriented
%  affine isomorphisms of $\bR$, $A_n \in \OriAffR$,
%  such that $A_n.F_n \tendsinlaw G$ where
%  $G$ is a nondegenerate c.d.f.
%
%  (TODO: The following is not quite correct. Need to assume one chooses
%  continuity points of the left-continuous inverse.)
%
%  Then for any $0 < p_1 < p_2 < 1$ and $0 < p_0 < 1$, defining
%  \begin{align*}
%    \tilde{a}_n = \frac{1}{\lcinv{F}_n(p_2) - \lcinv{F}_n(p_1)}
%    \quad \text{ and } \quad
%    \tilde{b}_n = \frac{\lcinv{F}_n(p_0)}{\lcinv{F}_n(p_2) - \lcinv{F}_n(p_1)}
%  \end{align*}
%  we have that
%  \begin{align*}
%    \widetilde{A}_n.F_n \tendsinlaw \widetilde{G}
%    \qquad \text{ where } \; \widetilde{A}_n(x) = \tilde{a}_n x + \tilde{b}_n .
%  \end{align*}
%\end{lemma}
\begin{proof}
  \uses{thm:convergence-to-types-different-affine}
  % \leanok
  \ldots
\end{proof}

\section{Self-similarity characterizations of the extreme value distributions}

\begin{lemma}[Continuous parameter extreme value limit relation]
  \label{lem:continuous-parameter-ev-limit}
  \uses{def:cdf, lem:oriented-affine-action-on-cdf}
  % \lean{}
  % \leanok
  Let $F$ be a c.d.f.

  \emph{(Note that below we use
  the sequence $(F^n)_{n \in \bN}$ of $n$th powers of a fixed c.d.f.,
  not a sequence of arbitrary c.d.f.s.
  Recall that the $n$th power $F^n$ is the c.d.f. of the maximum
  of $n$ independent random variables with the distribution $F$.)}

  Suppose that for a sequence $(A_n)_{n \in \bN}$ of oriented
  affine isomorphisms of $\bR$, $A_n \in \OriAffR$, we have
  \begin{align*}
    A_n.F^n \tendsinlaw G ,
  \end{align*}
  where $G$ is a c.d.f.

  Then, for any $t > 0$, denoting by $G^t$ the c.d.f. given by
  $G^t(x) = \big( G(x) \big)^t$, we have
  \begin{align*}
    A_{n}.F^{\lfloor n t \rfloor} \tendsinlaw G^t .
  \end{align*}
\end{lemma}

\begin{lemma}[Self-similarity of extreme value distributions]
  \label{lem:self-similarity-of-extreme-value-distributions}
  \uses{def:cdf, lem:oriented-affine-action-on-cdf, def:extr-val-distr}
  % \lean{}
  % \leanok
  %Let $F$ be a c.d.f.
  %
  %\emph{(Note that below we use
  %the sequence $(F^n)_{n \in \bN}$ of $n$th powers of a fixed c.d.f.,
  %not a sequence of arbitrary c.d.f.s.
  %Recall that the $n$th power $F^n$ is the c.d.f. of the maximum
  %of $n$ independent random variables with the distribution $F$.)}
  %
  %Suppose that for a sequence $(A_n)_{n \in \bN}$ of oriented
  %affine isomorphisms of $\bR$, $A_n \in \OriAffR$, we have
  %\begin{align*}
  %  A_n.F^n \tendsinlaw G ,
  %\end{align*}
  %where $G$ is a c.d.f.
  Suppose that $G$ is an extreme-value distribution.
  Then there exists a family $(A_t)_{t > 0}$ of
  oriented affine isomorphisms of $\bR$, $A_t \in \OriAffR$,
  such that for any $t > 0$
  \begin{align*}
    G^t = A_t . G .
  \end{align*}

  Moreover, $t \mapsto A_t$ is a measurable homomorphism
  of multiplicative groups $(0,+\infty) \to \OriAffR$.
\end{lemma}
\begin{proof}
  \uses{lem:continuous-parameter-ev-limit, thm:convergence-to-types}
  % \leanok
  \ldots
\end{proof}

%\begin{lemma}[Continuous parameter limit relation]
%  \label{lem:continuous-parameter-ev-limit}
%  \uses{def:cdf, lem:oriented-affine-action-on-cdf}
%  % \lean{}
%  % \leanok
%  Let $F$ be a c.d.f.
%
%  Suppose that for a sequence $(A_n)_{n \in \bN}$ of oriented
%  affine isomorphisms of $\bR$, and for a family $(G_t)_{t > 0}$
%  of c.d.f.s, \ldots.
%
%
%  and any $t > 0$, we have
%  we have
%  \begin{align*}
%    A_n.F^n \tendsinlaw G_t ,
%  \end{align*}
%  where $G_t$ is a nondegenerate c.d.f.
%
%  Then, for any $t > 0$, denoting by $G^t$ the c.d.f. given by
%  $G^t(x) = \big( G(x) \big)^t$, we have
%  \begin{align*}
%    A_{n}.F^{\lfloor n t \rfloor} \tendsinlaw G^t .
%  \end{align*}
%\end{lemma}

%%\begin{lemma}[Functional equation in one parameter subgroups of affine isomorphisms]
%%  \label{lem:affine-one-parameter-subgroup}
%%  \uses{def:lem:oriented-affine-action-on-cdf}
%%  % \lean{}
%%  % \leanok
%%  Suppose that $(A_t)_{t > 0}$ is a family of affine isomorphisms of $\bR$
%%  such that
%%  \begin{align*}
%%    A_{s t} = A_s \circ A_t  \qquad \text{ for any } s, t > 0 .
%%  \end{align*}
%%  Write $A_t(x) = a_t x + b_t$, with $a_t > 0$ and $b_t \in \bR$.
%%  Then we have, for any $s, t > 0$,
%%  \begin{align*}
%%    a_{t s} = \; & a_t \, a_s
%%      \qquad \text{ and } \\
%%    b_{t s} = \; & a_t \, b_s + b_t .
%%  \end{align*}
%%  (Also by symmetry $b_{t s} = a_s \, b_t + b_s$.)
%%\end{lemma}
%%\begin{proof}
%%  % \uses{}
%%  % \leanok
%%\end{proof}
%%
%%\begin{lemma}[Functional equation scaling coefficient solution]
%%  \label{lem:solution-functional-eqn-scaling}
%%  %\uses{def:lem:oriented-affine-action-on-cdf}
%%  % \lean{}
%%  % \leanok
%%  Suppose that $a \colon (0,+\infty) \to (0,+\infty)$ is a
%%  measurable function satisfying, for any $s, t > 0$,
%%  \begin{align*}
%%    a(t s) = \; & a(t) \, a(s) .
%%  \end{align*}
%%  Then there exists a $\theta \in \bR$ such that for all $t > 0$,
%%  \begin{align*}
%%    a(t) = t^{\theta} .
%%  \end{align*}
%%\end{lemma}
%%\begin{proof}
%%  % \uses{}
%%  % \leanok
%%\end{proof}
%%
%%\begin{lemma}[Functional equation translation coefficient solution with $\theta = 0$]
%%  \label{lem:solution-functional-eqn-translation-no-scaling}
%%  %\uses{def:lem:oriented-affine-action-on-cdf}
%%  % \lean{}
%%  % \leanok
%%  Suppose that
%%  $b \colon (0,+\infty) \to \bR$ is a measurable function
%%  satisfying, for any $s, t > 0$,
%%  \begin{align*}
%%    b(t s) = \; & b(s) + b(t) .
%%  \end{align*}
%%  Then there exists a constant $c$ such that for $t \in (0,+\infty)$
%%  we have
%%  \begin{align*}
%%    b(t) = \; & -c \, \log (t) .
%%  \end{align*}
%%\end{lemma}
%%\begin{proof}
%%  % \uses{}
%%  % \leanok
%%\end{proof}
%%
%%\begin{lemma}[Functional equation translation coefficient solution with $\theta \ne 0$]
%%  \label{lem:solution-functional-eqn-translation-with-scaling}
%%  %\uses{def:lem:oriented-affine-action-on-cdf}
%%  % \lean{}
%%  % \leanok
%%  Suppose that $\theta \in \bR \setminus 0$ and
%%  $b \colon (0,+\infty) \to \bR$ is a measurable function
%%  satisfying, for any $s, t > 0$,
%%  \begin{align*}
%%    b(t s) = \; & t^{\theta} \, b(s) + b(t) .
%%  \end{align*}
%%  Then there exists a constant $c$ such that for $t \in (0,+\infty) \setminus \set{1}$
%%  we have
%%  \begin{align*}
%%    b(t) = \; & c (1 - t^{-\theta}) .
%%  \end{align*}
%%\end{lemma}
%%\begin{proof}
%%  % \uses{}
%%  % \leanok
%%\end{proof}

\begin{lemma}[Self-similar continuous c.d.f. family characterization $\gamma = 0$]
  \label{lem:characterization-self-similar-family-zero-index}
  \uses{def:lem:oriented-affine-action-on-cdf}
  % \lean{}
  % \leanok
  Suppose that $G$ is a nondegenerate c.d.f. such that
  \begin{align*}
    G^{t} = A_t . G \qquad \text{ for any } t > 0 ,
  \end{align*}
  where
  \begin{align*}
    A_t(x) = x + \beta \, \log t
  \end{align*}
  with $\beta > 0$.

  Then with $d = \log \big(-\log G(0) \big)$,
  for all $x \in \bR$ we have
  \begin{align*}
    G(x) = \exp \Big( - \exp \big( -\beta^{-1} x + d \big) \Big) .
  \end{align*}
  (In particular, $G$ is of Gumbel type: there exists an $A \in \OriAffR$
  such that $G = A.\GumbelCDF$.)
\end{lemma}
\begin{proof}
  %\uses{lem:solution-functional-eqn-translation-no-scaling}
  % \leanok

  Since $G$ is nondegenerate, there exists an $x_0 \in \bR$
  with $0 < G(x_0) < 1$. Write $q = -\log G(x_0) > 0$,
  so that $G(x_0) = e^{-q}$.

  For $t > 0$, from the equation $G^{t} = A_t . G$
  we get for any $x \in \bR$
  \begin{align*}
  G(x)^t = G \big( A_t^{-1}(x) \big) = G \big( x - \beta \, \log(t) \big) .
  \end{align*}
  In particular, with $x = x_0 + \beta \, \log(t)$, we get
  \begin{align*}
    G\big( x_0 + \beta \, \log(t) \big)^{t} = G(x_0) = e^{- q},
  \end{align*}
  from which we can solve
  \begin{align*}
    G\big( x_0 + \beta \, \log(t) \big) = e^{-q/t} .
  \end{align*}
  The above holds for any $t>0$, and any $x \in \bR$
  can be written as $x = x_0 + \beta \, \log\big( e^{(x - x_0)/\beta} \big)$.
  We therefore get, for any $x \in \bR$,
  \begin{align*}
    G(x) = \exp \Big( - q \, \exp \big( -(x-x_0)/\beta \big) \Big) .
  \end{align*}
  This is of the desired form, with $d = \frac{x_0}{\beta} + \log(q)$.
  Plugging in $x = 0$ then shows $d = \log \big(-\log G(0) \big)$.
\end{proof}

\begin{lemma}[Self-similar continuous c.d.f. family characterization $\gamma > 0$]
  \label{lem:characterization-self-similar-family-pos-index}
  \uses{def:lem:oriented-affine-action-on-cdf}
  % \lean{}
  % \leanok
  Suppose that $G$ is a nondegenerate c.d.f. such that
  \begin{align*}
    G^{t} = A_t . G \qquad \text{ for any } t > 0 ,
  \end{align*}
  where
  \begin{align*}
    A_t(x) = t^{\alpha} x + c \, (1 - t^{\alpha}) = t^{\alpha} (x-c) + c
  \end{align*}
  with $c \in \bR$ and $\alpha > 0$.

  Then with $\sigma = \big(- \log G(c+1)\big)^{\alpha}$,
  for all $x \ge c$ we have
  \begin{align*}
    G(x) = \exp \Big( - \big( \frac{x - c}{\sigma} \big)^{-1/\alpha} \big) \Big) .
  \end{align*}

  (It easily follows that $G$ is of Fr\'echet type: there exists
  an $A \in \OriAffR$ such that $G = A.\FrechetCDF{\alpha}$.)
\end{lemma}
\begin{proof}
  %\uses{lem:solution-functional-eqn-translation-with-scaling}
  % \leanok

  Since $G$ is nondegenerate, there exists an $x_0 \in \bR$
  with $0 < G(x_0) < 1$. Write $q = -\log G(x_0) > 0$,
  so that $G(x_0) = e^{-q}$.

  %Note that
  %\begin{align*}
  %  A_t^{-1}(x) = t^{-\alpha} x + c (1 - t^{-\alpha}) = t^{-\alpha} (x-c) + c .
  %\end{align*}
  For $t > 0$, from the equation $G^{t} = A_t . G$
  we get for any $x \in \bR$
  \begin{align*}
  G(x)^t = G \big( A_t^{-1}(x) \big) . %= G \big( t^{-\alpha} (x-c) + c \big) .
  \end{align*}

  Note that for $x \le c$ and $t = 2$ we have $A_2^{-1}(x) = 2^{-\alpha}(x-c) + c \ge x$,
  so the above implies $G(x)^2 = G(A_2^{-1}(x)) \ge G(x)$.
  This not possible if $0 < G(x) < 1$, so we must in particular have $x_0 > c$.

  With $x = A_t(x_0)$ in the above equation, %= t^{\alpha} (x_0-c) + c$,
  we get
  \begin{align*}
    G\big( A_t(x_0) \big)^{t} = G(x_0) = e^{- q},
  \end{align*}
  from which we can solve
  \begin{align*}
    G(x) = G\big( A_t(x_0) \big) = e^{-q/t} .
  \end{align*}

  %The above holds for any $t>0$.
  Any $x \ge c$ can be written as $x = A_t(x_0) = t^{\alpha} (x_0-c) + c$
  with $t = \big( \frac{x - c}{x_0 - c} \big)^{1/\alpha}$
  (recall that $x_0 - c > 0$).
  We therefore get, for any $x \ge c$,
  \begin{align*}
    G(x) = \exp \Big( - q \, \big( \frac{x - c}{x_0 - c} \big)^{-1/\alpha} \big) \Big) .
  \end{align*}
  This is of the form
  \begin{align*}
    G(x) = \exp \Big( - \big( \frac{x - c}{\sigma} \big)^{-1/\alpha} \big) \Big) ,
  \end{align*}
  and plugging in $x = c + 1$ yields $\sigma = \big(- \log G(c+1)\big)^{\alpha}$.
  %This is of the desired form, with $d = \frac{x_0}{c} + \log(q)$.
  %Plugging in $x = 0$ then shows $d = \log \big(-\log G(0) \big)$.
\end{proof}

\begin{lemma}[Self-similar continuous c.d.f. family characterization $\gamma < 0$]
  \label{lem:characterization-self-similar-family-neg-index}
  \uses{def:lem:oriented-affine-action-on-cdf}
  % \lean{}
  % \leanok
  Suppose that $G$ is a nondegenerate c.d.f. such that
  \begin{align*}
    G^{t} = A_t . G \qquad \text{ for any } t > 0 ,
  \end{align*}
  where
  \begin{align*}
    A_t(x) = t^{-\alpha} x + c \, (1 - t^{-\alpha}) = t^{-\alpha} (x-c) + c
  \end{align*}
  with $c \in \bR$ and $\alpha > 0$.

  Then with $\sigma = \big(- \log G(c-1)\big)^{-\alpha}$,
  for all $x \le c$ we have
  \begin{align*}
    G(x) = \exp \Big( - \big( \frac{c - x}{\sigma} \big)^{1/\alpha} \big) \Big) .
  \end{align*}

  (It easily follows that $G$ is of Weibull type: there exists
  an $A \in \OriAffR$ such that $G = A.\WeibullCDF{\alpha}$.)
\end{lemma}
\begin{proof}
  %\uses{lem:solution-functional-eqn-scaling,
  %  lem:solution-functional-eqn-translation-exp-zero}
  % \leanok

  Since $G$ is nondegenerate, there exists an $x_0 \in \bR$
  with $0 < G(x_0) < 1$. Write $q = -\log G(x_0) > 0$,
  so that $G(x_0) = e^{-q}$.

  %Note that
  %\begin{align*}
  %  A_t^{-1}(x) = t^{-\alpha} x + c (1 - t^{-\alpha}) = t^{-\alpha} (x-c) + c .
  %\end{align*}
  For $t > 0$, from the equation $G^{t} = A_t . G$
  we get for any $x \in \bR$
  \begin{align*}
  G(x)^t = G \big( A_t^{-1}(x) \big) . %= G \big( t^{-\alpha} (x-c) + c \big) .
  \end{align*}

  Note that for $x \ge c$ and $t = 2$ we have $A_2^{-1}(x) = 2^{\alpha}(x-c) + c \ge x$,
  so the above implies $G(x)^2 = G(A_2^{-1}(x)) \ge G(x)$.
  This not possible if $0 < G(x) < 1$, so we must in particular have $x_0 < c$.

  With $x = A_t(x_0)$ in the above equation, %= t^{\alpha} (x_0-c) + c$,
  we get
  \begin{align*}
    G\big( A_t(x_0) \big)^{t} = G(x_0) = e^{- q},
  \end{align*}
  from which we can solve
  \begin{align*}
    G(x) = G\big( A_t(x_0) \big) = e^{-q/t} .
  \end{align*}

  %The above holds for any $t>0$.
  Any $x \le c$ can be written as $x = A_t(x_0) = t^{-\alpha} (x_0-c) + c$
  with $t = \big( \frac{c - x}{c - x_0} \big)^{-1/\alpha}$
  (recall that $c - x_0 > 0$).
  We therefore get, for any $x \le c$,
  \begin{align*}
    G(x) = \exp \Big( - q \, \big( \frac{c - x}{c - x_0} \big)^{1/\alpha} \big) \Big) .
  \end{align*}
  This is of the form
  \begin{align*}
    G(x) = \exp \Big( - \big( \frac{c - x}{\sigma} \big)^{1/\alpha} \big) \Big) ,
  \end{align*}
  and plugging in $x = c - 1$ yields $\sigma = \big(- \log G(c-1)\big)^{-\alpha}$.
  %This is of the desired form, with $d = \frac{x_0}{c} + \log(q)$.
  %Plugging in $x = 0$ then shows $d = \log \big(-\log G(0) \big)$.
\end{proof}


\begin{theorem}[Three types of extreme value distributions]
  \label{thm:three-types-of-extr-val-distr}
  \uses{def:extr-val-distr,
    def:std-Gumbel-cdf, def:std-Frechet-cdf, def:std-Weibull-cdf}
  % \lean{}
  % \leanok
  Any extreme value distribution $G$ is of one of the following types:
  \begin{description}
    \item{($\GumbelCDF$)} \ldots
    \item{($\FrechetCDF{{}}$)} \ldots
    \item{($\WeibullCDF{{}}$)} \ldots
  \end{description}
  In particular, $G$ has either the type of the Gumber c.d.f. $\GumbelCDF$,
  the Fr\'echet c.d.f. $\FrechetCDF{\alpha}$ for some $\alpha > 0$, or
  the Weibull c.d.f. $\WeibullCDF{\alpha}$ for some $\alpha > 0$.
\end{theorem}
\begin{proof}
  \uses{thm:one-parameter-subgroups-of-affine-isomorphisms,
    lem:self-similarity-of-extreme-value-distributions,
    lem:characterization-self-similar-family-zero-index,
    lem:characterization-self-similar-family-pos-index,
    lem:characterization-self-similar-family-neg-index}
  % \leanok
  \ldots
\end{proof}
