\chapter{Cumulative distribution functions}

\begin{definition}
  \label{def:cdf}
  \lean{CumulativeDistributionFunction}
  \leanok
  A function $F \colon \R \to \R$ is a
  \emph{cumulative distribution function} (c.d.f.) if
  \begin{enumerate}
  \item[(i)] $x \mapsto F(x)$ is increasing;
  \item[(ii)] $x \mapsto F(x)$ is right-continuous;
  \item[(iii)] $\lim_{x \to -\infty} F(x) = 0$ and $\lim_{x \to +\infty} F(x) = 1$.
  \end{enumerate}
\end{definition}

\begin{lemma}
  \label{lem:cdf-of-random-var}
  \uses{def:cdf}
  \lean{MeasureTheory.ProbabilityMeasure.cdf}
  \leanok
  If $X$ is a real-valued random variable, then the function
  $F \colon \R \to \R$ given by $F(x) = \PR \big[ X \le x \big]$ is a c.d.f.
\end{lemma}
\begin{proof}
  % \leanok
  Property (1.) in Definition~\ref{def:cdf} is obvious (by monotonicity of measures)
  and properties (2.) and (3.) are simple consequences of monotone convergence
  theorems for probability measures.
\end{proof}

\section{Degenerate distributions}

\begin{definition}
  \label{def:degenerate-cdf}
  \uses{def:cdf}
  \lean{CumulativeDistributionFunction.IsDegenerate}
  \leanok
  A c.d.f. $F$ is said to be
  \emph{degenerate} if for every $x \in \R$ we have either $F(x) = 0$ or $F(x) = 1$.
  Otherwise $F$ is said to be \emph{nondegenerate}.
\end{definition}

\begin{lemma}
  \label{lem:degenerate-cdf-iff-exists-jump}
  \uses{def:degenerate-cdf}
  \lean{CumulativeDistributionFunction.isDegenerate_iff}
  \leanok
  $F$ is a degenerate c.d.f. if and only if there exists a $x_0 \in \R$ such that
  \begin{align*}
  F(x) = \begin{cases}
          0 & \text{ for } x < x_0 \\
          1 & \text{ for } x \ge x_0 .
         \end{cases}
  \end{align*}
\end{lemma}
\begin{proof}
  \uses{}
  % \leanok
  The ``if'' direction is clear. To prove the ``only if'' direction, assume that
  $F$ is a degenerated c.d.f., and let $x_0 = \inf \set{ x \in \R \; \big| \; F(x) = 1}$.
  Then it is straightforward to show by properties of a c.d.f. that $F$ has the asserted form.
\end{proof}

\begin{lemma}
  \label{lem:delta-has-degenerate-cdf}
  \uses{def:degenerate-cdf}
  \lean{CumulativeDistributionFunction.diracProba_is_degenerate}
  \leanok
  The c.d.f. of Dirac delta mass $\delta_{x_0}$ at $x_0 \in \R$ is degenerate.
\end{lemma}
\begin{proof}
  \uses{lem:degenerate-cdf-iff-exists-jump}
  % \leanok
  \ldots
\end{proof}

\begin{lemma}
  \label{lem:degenerate-cdf-is-delta}
  \uses{def:degenerate-cdf}
  \lean{CumulativeDistributionFunction.eq_diracProba_of_isDegenerate}
  % \leanok
  If a c.d.f. $F$ is degenerate, then it is the c.d.f. of a Dirac delta mass $\delta_{x_0}$
  at some point $x_0 \in \R$.
\end{lemma}
\begin{proof}
  \uses{lem:degenerate-cdf-iff-exists-jump}
  % \leanok
  \ldots
\end{proof}

\section{Distributions of maxima of independent random variables}

% \begin{lemma}
% \label{lem:random-var-of-cdf}
% \uses{def:cdf}
% If $F \colon \R \to \R$ is a c.d.f., then there exists a probability
% space $(\Omega, \mathcal{F}, \PR)$ and a real-valued random variable
% $X \colon \Omega \to \R$ such that for every $x \in \R$
% we have $F(x) = \PR \big[ X \le x \big]$.
% \end{lemma}
% \begin{proof}
% Abstract nonsense (take $\Omega = [0,1]$ and $X = \idOf{\R}$ ).
% \end{proof}

\begin{lemma}
  \label{lem:cdf-of-max-two}
  \uses{def:cdf}
  % \leanok
  Let $X$ and $Y$ be two independent real-valued random variables with respective
  cumulative distribution functions $F$ and $G$,
  i.e. $F(x) = \PR \big[ X \le x \big]$ and $G(x) = \PR \big[ Y \le x \big]$.
  Then the c.d.f. of $M = \max (X, Y)$ is $x \mapsto F(x) \, G(x)$.
\end{lemma}
\begin{proof}
  \uses{}
  % \leanok
  Fix $x \in \R$. Note that $\max (X, Y) \le x$ if and only if both $X \le x$ and $Y \le x$.
  Calculate, using independence,
  \begin{align*}
  \PR\big[ \max (X, Y) \le x \big]
  \; = \; \PR\big[ X \le x \, \; Y \le x \big]
  \; = \; \PR\big[ X \le x \big] \; \PR\big[ Y \le x \big]
  \; = \; F(x) \, G(x) .
  \end{align*}
\end{proof}

\begin{lemma}
  \label{lem:cdf-of-max-many}
  \uses{def:cdf}
  % \leanok
  Let $X_0, X_1, \ldots, X_{n-1}$ be independent identically distributed real-valued random variables
  with cumulative distribution functions $F$,
  i.e. $F(x) = \PR \big[ X_j \le x \big]$ for every $j$.
  Then the c.d.f. of \[ M_n = \max_{0 \le j < n} X_j\] is the function $x \mapsto \big(F(x)\big)^n$.
\end{lemma}
\begin{proof}
  \uses{lem:cdf-of-max-two}
  % \leanok
  Induction on~$n$ using \ref{lem:cdf-of-max-two}.
\end{proof}

\begin{lemma}
  \label{lem:cdf-of-affine-max-many}
  \uses{def:cdf}
  % \leanok
  Let $X_0, X_1, \ldots, X_{n-1}$ be independent identically distributed real-valued random variables
  with cumulative distribution functions $F$,
  i.e. $F(x) = \PR \big[ X_j \le x \big]$ for every $j$,
  and let $a > 0$ and $b \in \R$.
  Then the c.d.f. of
  \begin{align*}
  \hat{M}_n = \frac{\max_{0 \le j < n} X_j - b}{a}
  \end{align*}
  is the function $x \mapsto \big(F(a x + b)\big)^n$.
\end{lemma}
\begin{proof}
  \uses{lem:cdf-of-max-many}
  % \leanok
  Use \ref{lem:cdf-of-max-many} and do a change of variables.
\end{proof}

\section{Distributions of minima of independent random variables}

\section{Equivalence classes modulo affine transformations}

\begin{definition}
  \label{def:oriented-affine-isomorphism}
  \lean{orientationPreservingAffineEquiv}
  \leanok
  The collection of all transformations $\R \to \R$ of the form
  $x \mapsto a x + b$, where $a>0$, $b \in \R$, forms a group.
  We call this the \emph{orientation preserving affine isomorphism group}
  and denote it by $\OriAffR$.
\end{definition}

\begin{definition}
  \label{def:oriented-affine-transform-of-cdf}
  \uses{def:oriented-affine-isomorphism, def:cdf}
  \lean{CumulativeDistributionFunction.affineTransform}
  \leanok
  The action of an orientation preserving affine isomorphism $A \in \OriAffR$
  on a cumulative distribution function $F$ is defined so
  that $A.F \colon \R \to \R$ is given by $(A.F)(x) = F(A^{-1}(x))$.
  Then $A.F$ is also a c.d.f.
\end{definition}

\begin{lemma}
  \label{lem:oriented-affine-action-on-cdf}
  \uses{def:oriented-affine-transform-of-cdf}
  \lean{CumulativeDistributionFunction.instMulActionOrientationPreservingAffineEquiv}
  \leanok
  The actions of orientation preserving affine isomorphisms
  on a cumulative distribution functions is a group action, i.e.,
  $1.F = F$ and $(AB).F = A.(B.F)$ for any c.d.f. $F$ and any $A,B \in \OriAffR$.
\end{lemma}
\begin{proof}
  \uses{}
  % \leanok
  Direct calculations.
\end{proof}

\begin{lemma}
  \label{lem:degenerate-cdf-transform}
  \uses{lem:oriented-affine-action-on-cdf, def:degenerate-cdf}
  \lean{CumulativeDistributionFunction.affine_isDegenerate_iff}
  \leanok
  Let $F$ be a cumulative distribution function
  and $A \in \OriAffR$ an orientation preserving affine isomorphism.
  Then $A.F$ is degenerate if and only if $F$ is degenerate.
\end{lemma}
\begin{proof}
  \uses{}
  \leanok
  Straightforward from the definitions.
\end{proof}

\begin{lemma}
  \label{lem:cdf-continuity-pt-transform}
  \uses{lem:oriented-affine-action-on-cdf}
  \lean{CumulativeDistributionFunction.affine_continuousAt_of_continuousAt}
  \leanok
  Let $F$ be a cumulative distribution function,
  and $A \in \OriAffR$ an orientation preserving affine isomorphism.
  If a point $x \in \R$ is a continuity point of $F$, then
  the point $A(x) \in \R$ is a continuity point of $A.F$.
\end{lemma}
\begin{proof}
  \uses{}
  % \leanok
  Straightforward.
\end{proof}


% \begin{lemma}
% \label{lem:cdf-affine-group-action}
% \uses{def:cdf}
% Suppose $a > 0$ and $b \in \R$. Then for any c.d.f. $F$, also
% the function $x \mapsto F(a x + b)$ is a c.d.f.
% \end{lemma}
% \begin{proof}
% \ldots
% \end{proof}
%
% \begin{definition}
% \label{def:cdf}
% %\lean{CumulativeDistributionFunction}
% %\uses{def:degenerate-cdf}
% \uses{def:cdf}
% A function $F \colon \R \to \R$ is a
% \emph{cumulative distribution function} (c.d.f.) if
% \begin{enumerate}
% \item[(i)] $x \mapsto F(x)$ is increasing;
% \item[(ii)] $x \mapsto F(x)$ is right-continuous;
% \item[(iii)] $\lim_{x \to -\infty} F(x) = 0$ and $\lim_{x \to +\infty} F(x) = 1$.
% \end{enumerate}
% \end{definition}
%

\begin{lemma}[Continuity points of c.d.f.s are those which carry no point mass]
  \label{lem:cdf-continuity-pt-iff-measure-singleton}
  \uses{def:cdf}
  \lean{CumulativeDistributionFunction.continuousAt_iff}
  \leanok
  Let $F$ be cumulative distribution function of a
  probability measure $\PRmeas$ on $\bR$. A point $x \in \bR$ is
  a continuity point of $F$ if and only if $\PRmeas[\set{x}] = 0$.
\end{lemma}
\begin{proof}
  % \uses{}
  % \leanok
  A c.d.f. is always continuous from the right.

  Continuity of $F$ from the left at $x$ means that for any
  sequence $(x_n)_{n \in \bN}$ increasing to $x$
  (i.e., $x_n \le x_{n+1} < x$ for all $n \in \bN$)
  \begin{align*}
    F(x_n) \to F(x) ,
  \end{align*}
  or equivalently in terms of measures
  \begin{align*}
    \PRmeas \big[ (-\infty,x_n] \big] \to \PRmeas \big[ (-\infty,x_n] \big] .
  \end{align*}
  But by monotone convergence of measures, we always have
  \begin{align*}
    \PRmeas \big[ (-\infty,x_n] \big]
    \to \; & \PRmeas \big[ (-\infty,x_n) \big] \\
    = \; & \PRmeas \big[ (-\infty,x_n] \big] - \PRmeas[\set{x}] .
  \end{align*}
  A comparison of these conditions shows that $F$ is also continuous from the left
  at $x$ if and only if $\PRmeas[\set{x}] = 0$.
\end{proof}
