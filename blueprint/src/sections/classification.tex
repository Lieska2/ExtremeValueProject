\chapter{Classification of extreme value distributions}

\section{Auxiliary classification I}

\begin{lemma}[Second order differential equation for Q]
  \label{lem:ode-of-order-two-for-Q}
  %\uses{}
  %\lean{}
  % \leanok
  Suppose that
  \begin{align*}
  Q \colon \bR \to \bR
  \end{align*}
  is differentiable and satisfies
  \begin{align*}
  Q(0) = 0 \qquad \text{ and } \qquad Q'(0) = 1
  \end{align*}
  and
  \begin{align*}
  Q(h+s) = Q(h) \alpha(s) + Q(s)
  \end{align*}
  for some $\alpha \colon \bR \to \bR$ and every $s,h \in \bR$.
  Then $Q$ is twice continuously differentiable and satisfies
  \begin{align}
    Q'' (s) = Q'(s) \, Q'' (0) \qquad \text{ for every } s \in \bR .
  \end{align}
\end{lemma}
\begin{proof}
  %\uses{}
  % \leanok
  Note first that the equation implies (rearranging and dividing by $h$),
  for any $s$ and $h \ne 0$,
  \begin{align*}
    \frac{Q(h+s) - Q(s)}{h} = \frac{Q(h)}{h} \alpha(s) .
  \end{align*}
  Taking the limit as $h \to 0$ and using $Q(0)=0$ yields
  $Q'(s) = Q'(0) \, \alpha(s) = \alpha(s)$, where we also
  took into account $Q'(0) = 1$. Therefore necessarily
  $\alpha = Q'$, and the equation can be rewritten in the form
  \begin{align*}
  Q(h+s) = Q(h) \, Q'(s) + Q(s) \qquad \text{ for } s,h \in \bR .
  \end{align*}

  Rearranging the equation, we find
  \begin{align*}
    Q(h+s) - Q(s) = Q(h) \, Q'(s)
  \end{align*}
  and interchanging the role of $s$ and $h$ also
  \begin{align*}
    Q(h+s) - Q(h) = Q(s) \, Q'(h) .
  \end{align*}
  Subtracting the last two equations yields
  \begin{align*}
    Q(s) - Q(h) = Q(s) \, Q'(h) - Q(h) \, Q'(s) ,
  \end{align*}
  which by rearranging and dividing by $h \ne 0$ yields
  \begin{align*}
     \frac{Q(h)}{h} \big( Q'(s) - 1 \big) = Q(s) \frac{Q'(h) - 1}{h} .
  \end{align*}
  Taking the limit $h \to 0$, recalling $Q(0)=1$ and $Q'(0)=1$,
  gives the existence of the second derivative $Q''(0)$ and the equation
  \begin{align*}
    Q'(s) - 1 = Q(s) Q''(0) .
  \end{align*}
  Solving $Q'(s) = 1 + Q''(0) \, Q(s)$ and recalling that $Q$
  is differentiable shows that $Q'$ is also differentiable, so
  $Q$ is indeed twice differentiable. Differentiating, we get
  the asserted equation
  \begin{align*}
    Q''(s) = Q'(s) \, Q''(0) .
  \end{align*}
  Since $Q'$ is differentiable and in particular continuous, this also shows
  that $Q''$ is continuous, i.e., that $Q$ is twice continuously differentiable.
\end{proof}

\begin{lemma}[Solution for Q]
  \label{lem:solve-Q}
  %\uses{}
  %\lean{}
  % \leanok
  Suppose that $Q \colon \bR \to \bR$ is twice continuously differentiable
  and $Q'$ is positive and $Q$
  and satisfies $Q(0)=0$, $Q'(0) = 1$, and the equation concluded in
  Lemma~\ref{lem:ode-of-order-two-for-Q} with $\gamma = Q''(0)$, i.e.,
  \begin{align}\label{eq: Q eqn dd}
  Q'' (s) = \gamma \, Q'(s) \qquad \text{ for every } s \in \bR .
  \end{align}
  Then $Q$ is given by
  \begin{align*}
  Q(s) = \begin{cases}
    \frac{e^{\gamma s} - 1}{\gamma} & \; \text{ if $\gamma \ne 0$} \\
    \;\; s & \; \text{ if $\gamma = 0$}
    \end{cases}
    \qquad \text{for $s \in \bR$.}
  \end{align*}
%  if $\gamma \ne 0$ and
%  \begin{align*}
%  Q(s) = s
%  \end{align*}
%  if $\gamma = 0$.
  \end{lemma}
\begin{proof}
  %\uses{}
  % \leanok
  Since $Q$ is differentiable and $Q'(s)>0$ for any $s \in \bR$,
  we can write \eqref{eq: Q eqn dd} as
  \begin{align*}
  \der{s} \log Q'(s)
    = \frac{Q''(s)}{Q'(s)}
    = \gamma .
  \end{align*}
  Integrating and noting $\log Q'(0) = \log 1 = 0$, this
  yields $\log Q'(s) = \gamma s$ for all $s \in \bR$, or equivalently
  $Q'(s) = e^{\gamma s}$. Integrating a second time
  and noting $Q(0) = 0$, this yields the desired formulas
  in the two cases $\gamma \ne 0$ and $\gamma = 0$, respectively, as follows.

  If $\gamma \ne 0$, then we get
  \begin{align*}
  Q(s) = \int_0^s e^{\gamma u} \, \ud u = \frac{e^{\gamma s} - 1}{\gamma} .
  \end{align*}

  If $\gamma = 0$, then we get
  \begin{align*}
  Q(s) = \int_0^s e^0 \, \ud u = s .
  \end{align*}
\end{proof}

% TODO: Move to a different file.
\begin{theorem}[Monotone functions are a.e. differentiable]
  \label{thm:monotone-ae-differentiable}
  %\uses{}
  \lean{Monotone.ae_differentiableAt}
  \leanok
  \mathlibok
  If $f : \bR \to \bR$ is nondecreasing, then the derivative
  $f'(x)$ exists at almost every $x \in \bR$.
  In particular there exists points $x$ where $f'(x)$ exists.
\end{theorem}
\begin{proof}
  %\uses{}
  \leanok
  (The proof is already in Mathlib.)
\end{proof}

\begin{lemma}[Solution for E]
  \label{lem:solve-E}
  %\uses{lem:solve-Q}
  %\lean{}
  % \leanok
  Suppose that $E \colon (0,\infty) \to \bR$
  is nondecreasing and nonconstant function which satisfies $E(1) = 0$ and
  \begin{align*}
  E(\lambda \sigma) = E(\lambda) A(\sigma) + E(\sigma)
  \end{align*}
  for some $A \colon (0,\infty) \to (0,\infty)$ and all $\lambda, \sigma > 0$.
  Then, denoting $c = E'(1)$ and $\gamma = \frac{1}{E'(1)} \dder{s} E(e^s) \big|_{s=0}$,
  for all $\lambda \in \bR$ we have
  \begin{align*}
  E(\lambda) = \begin{cases}
    c \big( \lambda^\gamma - 1 \big) & \;\;\text{ if $\gamma \ne 0$.} \\
    c \, \log (\lambda) & \;\;\text{ if $\gamma = 0$.}
    \end{cases}
  \end{align*}
\end{lemma}
\begin{proof}
  \uses{lem:ode-of-order-two-for-Q, lem:solve-Q, thm:monotone-ae-differentiable}
  % \leanok
  Denote $H(s) = E(e^s)$ for $s \in \bR$. Then $H(0) = E(1) = 0$ and
  $H$ is also nondecreasing and nonconstant.
  By Theorem~\ref{thm:monotone-ae-differentiable}, there
  exists some $s_0 \in \bR$ such that the derivative $H'(s_0)$ exists.

  The equation for $E$ yields an equation for $H$,
  \begin{align*}
  H(h+s) = H(h) A(e^{s}) + H(s) \qquad \text{ for any } s, h \in \bR.
  \end{align*}
  We can rearrange this equation and divide by $h$,
  and we get for any $s, h \in \bR$, $h \ne 0$,
  \begin{align*}
  \frac{H(h+s) - H(s)}{h} = \frac{H(h)}{h} A(e^{s}) .
  \end{align*}
  If $s = s_0$, then the LHS tends to $H'(s_0)$ as $h \to 0$.
  The RHS must therefore also have a limit as $s \to 0$, and
  observing that $\frac{H(h)}{h} = \frac{H(h) - H(0)}{h}$,
  that limit is $H'(0) A(e^{s_0})$, which shows that the derivative $H'(0)$ exists.
  Then applying the same equation at general $s \in \bR$ shows that
  $H'(s) = H'(0) A(e^{s})$ must exist, so $H \colon \bR \to \bR$ is
  in fact everywhere differentiable.

  Note that we must have $H'(0) > 0$, because if $H'(0) = 0$
  then by the above equation $H'(s) = 0$ for every $s$ and then
  $H$ is a constant function, which is a contradiction.
  Denote $c = H'(0) = \der{s} E(e^s) \big|_{s=0} = E'(1)$.

  Also the equation and positivity of the function $A$ give $H'(s)>0$
  for all $s$. %, so $H$ is strictly increasing.

  Now consider $Q(s) = \frac{H(s)}{c}$. This $H$ is
  obviously also differentiable, since $H$ is. The equation
  for $H$ yields the following equation for $Q$,
  \begin{align*}
    Q(h+s) = Q(h) \alpha(s) + Q(s) \qquad \text{ for any } s,h \in \bR ,
  \end{align*}
  where $\alpha(s) = A(e^{s})$.
  By Lemma~\ref{lem:ode-of-order-two-for-Q} $Q$ is then twice
  continuously differentiable and satisfies
  \begin{align}
  Q'(s) Q'' (0) = Q'' (s) \qquad \text{ for all } s \in \bR .
  \end{align}
  By Lemma~\ref{lem:solve-Q} we then get
  a formula for $Q$, which involves
  \begin{align*}
  \gamma = Q''(0) = \frac{H''(0)}{c}
    = \frac{1}{c} \dder{s} E(e^s) \big|_{s = 0} .
  \end{align*}

  If $\gamma \ne 0$ then the formula reads
  $Q(s) = \frac{e^{\gamma s} - 1}{\gamma}$.
  Now just tracing the definitions, we get the asserted formula
  \begin{align*}
  E(\lambda) = H(\log \lambda)
    = c \, Q(\log \lambda) = c \, \big( e^{\gamma \, \log(\lambda)} - 1 \big)
    = c \, \big( \lambda^{\gamma} - 1 \big) .
  \end{align*}

  If $\gamma = 0$ then the formula reads $Q(s) = s$.
  Now just tracing the definitions, we get instead
  \begin{align*}
  E(\lambda) = H(\log \lambda)
    = c \, Q(\log \lambda) = c \, \log(\lambda) ,
  \end{align*}
  and the proof is complete.
\end{proof}


\section{Inverting to recover a cumulative distribution function}

%\begin{lemma}[Classification of cdfs of evds]
%  \label{lem:cdf-of-evd-classification}
%  %\uses{}
%  %\lean{}
%  % \leanok
%  Suppose that $G \colon \bR \to [0,1]$ is a c.d.f.
%  whose right-continuous inverse is given by $G^\leftarrow = \cdots$
%\end{lemma}
