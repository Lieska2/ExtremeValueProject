% In this file you should put the actual content of the blueprint.
% It will be used both by the web and the print version.
% It should *not* include the \begin{document}
%
% If you want to split the blueprint content into several files then
% the current file can be a simple sequence of \input. Otherwise It
% can start with a \section or \chapter for instance.

\chapter{Extreme value distributions}

\section{Definition of extreme value distributions}

Lemma~\ref{lem:cdf-of-affine-max-many} motivates the following definition.

\begin{definition}
  \label{def:extr-val-distr}
  \uses{lem:oriented-affine-action-on-cdf}
  \lean{CumulativeDistributionFunction.IsExtremeValueDistr}
  % \leanok
  A c.d.f. $G$ is said to be an \emph{extreme value distribution} if
  $G$ is nondegenerate and
  there exists a c.d.f. $F$ and a sequence $(A_n)_{n \in \N}$ of orientation
  preserving affine isomorphisms $A_n \in \OriAffR$, such that
  for every continuity point $x \in \R$ of~$G$ we have
  \begin{align*}
  \lim_{n \to \infty} \big( (A_n.F)(x) \big)^n \; = \; G(x) .
  \end{align*}
\end{definition}

\begin{lemma}
  \label{lem:extr-val-distr-transform}
  \uses{lem:oriented-affine-action-on-cdf, def:extr-val-distr}
  \lean{CumulativeDistributionFunction.IsExtremeValueDistr.affineTransform}
  \leanok
  Let $G$ be an extreme value distribution and
  and $A \in \OriAffR$ an orientation preserving affine isomorphism.
  Then also $A.G$ is an extreme value distribution.
\end{lemma}
\begin{proof}
  \uses{lem:cdf-continuity-pt-transform, lem:degenerate-cdf-transform}
  % \leanok
  Straightforward using Lemmas~\ref{lem:cdf-continuity-pt-transform}
  and~\ref{lem:degenerate-cdf-transform}.
\end{proof}



\section{Three types of extreme value distributions}

\begin{definition}
  \label{def:std-Gumbel-cdf}
  \uses{def:cdf}
  \lean{standardGumbelCDF}
  % \leanok
  \emph{The standard Gumbel distribution} is the c.d.f. $F_\Gumbel$ given by
  \begin{align*}
  F_\Gumbel (x) = \exp\big(-\exp(-x)\big) .
  \end{align*}
\end{definition}

\begin{definition}
  \label{def:std-Weibull-cdf}
  \uses{def:cdf}
  \lean{standardWeibullCDF}
  % \leanok
  \emph{The standard Weibull distribution} is the c.d.f. $F_\Weibull$ given by
  \begin{align*}
  F_\Weibull (x) = \cdots .
  \end{align*}
\end{definition}

\begin{definition}
  \label{def:std-Frechet-cdf}
  \uses{def:cdf}
  \lean{standardFrechetCDF}
  % \leanok
  \emph{The standard Fr\'echet distribution} is the c.d.f. $F_\Frechet$ given by
  \begin{align*}
  F_\Frechet (x) = \cdots .
  \end{align*}
\end{definition}

\begin{theorem}
  \label{thm:Gumbel-is-extr-val-distr}
  \uses{def:std-Gumbel-cdf, def:extr-val-distr}
  \lean{isExtremeValueDistr_standardGumbelCDF}
  % \leanok
  The standard Gumbel distribution $F_\Gumbel$ is an extreme value
  distribution.
\end{theorem}
\begin{proof}
  %\uses{lem:cdf-of-max-many}
  % \leanok
  Set $A_n(x) = x - \log(n)$ for $n \in \bN$. Then
  $A_n^{-1}(x) = x + \log(n)$ and for any $n \ge 1$ and $x \in \bR$ we get
  \begin{align*}
    \big( (A_n . F_\Gumbel)(x) \big)^n
    = \; & \big( (F_\Gumbel)(x + \log(n)) \big)^n \\
    = \; & \Big( \exp \big( - \exp(-(x + \log n)) \big) \Big)^n \\
    = \; & \Big( \exp \big( - n \exp(-x - \log n) \big) \Big) \\
    = \; & \Big( \exp \big( - n \, e^{-x} \, e^{-\log n}  \big) \Big) \\
    = \; & \Big( \exp \big( - n \, e^{-x} \, n^{-1} \big) \Big) \\
    = \; & \Big( \exp \big( - e^{-x} \big) \Big)
    = F_\Gumbel(x) .
  \end{align*}
  Since the above is true for each $n$, we in particular have
  \begin{align*}
    \lim_{n \to \infty} \big( (A_n . F_\Gumbel)(x) \big)^n
    = F_\Gumbel(x)
  \end{align*}
  for all $x \in \bR$. Since $F_\Gumbel$ is also nondegenerate,
  this shows that it is an extreme value distribution.
\end{proof}

\begin{theorem}
  \label{thm:Weibull-is-extr-val-distr}
  \uses{def:std-Weibull-cdf, def:extr-val-distr}
  \lean{isExtremeValueDistr_standardWeibullCDF}
  % \leanok
  The standard Weibull distribution $F_\Weibull$ is an extreme value
  distribution.
\end{theorem}
\begin{proof}
  \uses{lem:cdf-of-max-many}
  % \leanok
  \ldots
\end{proof}

\begin{theorem}
  \label{thm:Frechet-is-extr-val-distr}
  \uses{def:std-Frechet-cdf, def:extr-val-distr}
  \lean{isExtremeValueDistr_standardFrechetCDF}
  % \leanok
  The standard Fr\'echet distribution $F_\Frechet$ is an extreme value
  distribution.
\end{theorem}
\begin{proof}
  \uses{}
  % \leanok
  \ldots
\end{proof}
