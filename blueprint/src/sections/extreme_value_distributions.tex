% In this file you should put the actual content of the blueprint.
% It will be used both by the web and the print version.
% It should *not* include the \begin{document}
%
% If you want to split the blueprint content into several files then
% the current file can be a simple sequence of \input. Otherwise It
% can start with a \section or \chapter for instance.

\chapter{Extreme value distributions}

\section{Definition of extreme value distributions}

Lemma~\ref{lem:cdf-of-affine-max-many} motivates the following definition.

\begin{definition}
  \label{def:extr-val-distr}
  \uses{lem:oriented-affine-action-on-cdf}
  \lean{CumulativeDistributionFunction.IsExtremeValueDistr}
  \leanok
  A c.d.f. $G$ is said to be an \emph{extreme value distribution} if
  $G$ is nondegenerate and
  there exists a c.d.f. $F$ and a sequence $(A_n)_{n \in \N}$ of orientation
  preserving affine isomorphisms $A_n \in \OriAffR$, such that
  for every continuity point $x \in \R$ of~$G$ we have
  \begin{align*}
  \lim_{n \to \infty} \big( (A_n.F)(x) \big)^n \; = \; G(x) .
  \end{align*}
\end{definition}

\begin{lemma}
  \label{lem:extr-val-distr-transform}
  \uses{lem:oriented-affine-action-on-cdf, def:extr-val-distr}
  \lean{CumulativeDistributionFunction.IsExtremeValueDistr.affineTransform}
  \leanok
  Let $G$ be an extreme value distribution and
  and $A \in \OriAffR$ an orientation preserving affine isomorphism.
  Then also $A.G$ is an extreme value distribution.
\end{lemma}
\begin{proof}
  \uses{lem:cdf-continuity-pt-transform, lem:degenerate-cdf-transform}
  % \leanok
  Straightforward using Lemmas~\ref{lem:cdf-continuity-pt-transform}
  and~\ref{lem:degenerate-cdf-transform}.
\end{proof}



\section{Three types of extreme value distributions}

\begin{definition}
  \label{def:std-Gumbel-cdf}
  \uses{def:cdf}
  \lean{standardGumbelCDF}
  % \leanok
  \emph{The standard Gumbel distribution} is the c.d.f. $F_\Gumbel$ given by
  \begin{align*}
  F_\Gumbel (x) = \exp\big(-\exp(-x)\big) .
  \end{align*}
\end{definition}

\begin{definition}
  \label{def:std-Weibull-cdf}
  \uses{def:cdf}
  \lean{standardWeibullCDF}
  % \leanok
  \emph{The standard Weibull distribution} is the c.d.f. $F_\Weibull$ given by
  \begin{align*}
  F_\Weibull (x) = \cdots .
  \end{align*}
\end{definition}

\begin{definition}
  \label{def:std-Frechet-cdf}
  \uses{def:cdf}
  \lean{standardFrechetCDF}
  % \leanok
  \emph{The standard Fr\'echet distribution} is the c.d.f. $F_\Frechet$ given by
  \begin{align*}
  F_\Frechet (x) = \cdots .
  \end{align*}
\end{definition}

\begin{theorem}
  \label{thm:Gumbel-is-extr-val-distr}
  \uses{def:std-Gumbel-cdf, def:extr-val-distr}
  \lean{isExtremeValueDistr_standardGumbelCDF}
  \leanok
  The standard Gumbel distribution $F_\Gumbel$ is an extreme value
  distribution.
\end{theorem}
\begin{proof}
  %\uses{lem:cdf-of-max-many}
  % \leanok
  Set $A_n(x) = x - \log(n)$ for $n \in \bN$. Then
  $A_n^{-1}(x) = x + \log(n)$ and for any $n \ge 1$ and $x \in \bR$ we get
  \begin{align*}
    \big( (A_n . F_\Gumbel)(x) \big)^n
    = \; & \big( (F_\Gumbel)(x + \log(n)) \big)^n \\
    = \; & \Big( \exp \big( - \exp(-(x + \log n)) \big) \Big)^n \\
    = \; & \exp \big( - n \exp(-x - \log n) \big) \\
    = \; & \exp \big( - n \, e^{-x} \, e^{-\log n}  \big) \\
    %= \; & \exp \big( - n \, e^{-x} \, n^{-1} \big) \\
    = \; & \exp \big( - e^{-x} \big) \\
    = \; & F_\Gumbel(x) .
  \end{align*}
  Since the above is true for each $n$, we in particular have
  \begin{align*}
    \lim_{n \to \infty} \big( (A_n . F_\Gumbel)(x) \big)^n
    = F_\Gumbel(x)
  \end{align*}
  for all $x \in \bR$. Since $F_\Gumbel$ is also nondegenerate,
  this shows that it is an extreme value distribution.
\end{proof}

\begin{theorem}
  \label{thm:Weibull-is-extr-val-distr}
  \uses{def:std-Weibull-cdf, def:extr-val-distr}
  \lean{isExtremeValueDistr_standardWeibullCDF}
  % \leanok
  The standard Weibull distribution $F_\Weibull$ is an extreme value
  distribution.
\end{theorem}
\begin{proof}
  %\uses{lem:cdf-of-max-many}
  % \leanok
  \ldots
\end{proof}

\begin{theorem}
  \label{thm:Frechet-is-extr-val-distr}
  \uses{def:std-Frechet-cdf, def:extr-val-distr}
  \lean{isExtremeValueDistr_standardFrechetCDF}
  % \leanok
  The standard Fr\'echet distribution $F_\Frechet$ is an extreme value
  distribution.
\end{theorem}
\begin{proof}
  \uses{}
  % \leanok
  \ldots
\end{proof}



\section{Equivalent formulations of the limit relation}

\begin{lemma}[Logarithmic version of the limit relation]
  \label{lem:log-ev-limit}
  \uses{def:cdf, lem:oriented-affine-action-on-cdf}
  \lean{ev_limit_iff_log_ev_limit}
  \leanok
  Let $F$ and $G$ be c.d.f.s, and $(A_n)_{n \in \N}$ a sequence of orientation
  preserving affine isomorphisms $A_n \in \OriAffR$. Then for any $x \in \bR$
  such that $0 < G(x) < 1$, the two conditions
  \begin{align*}
  \text{(i)} & & \lim_{n \to \infty} \big( (A_n.F)(x) \big)^n \, = \; & G(x) \\
  \text{(ii)} & & \lim_{n \to \infty} n \, \log F (A_n^{-1}(x)) \, = \; & \log G(x)
  \end{align*}
  are equivalent.
\end{lemma}
\begin{proof}
  % \uses{}
  % \leanok
  Recall that $(A_n.F)(x) = F (A_n^{-1}(x))$.
  Then just take logarithms (and use continuity) to get from (i) to (ii),
  and take exponentials (and use continuity) to get from (ii) to (i).
\end{proof}

\begin{lemma}[Relation implies $F$ tending to one]
  \label{lem:ev-limit-cdf-affine-tendsto-one}
  \uses{def:cdf, lem:oriented-affine-action-on-cdf}
  \lean{tendsto_one_of_ev_limit}
  \leanok
  Let $F$ and $G$ be c.d.f.s, and $(A_n)_{n \in \N}$ a sequence of orientation
  preserving affine isomorphisms $A_n \in \OriAffR$. Then for any $x \in \bR$
  such that $0 < G(x) < 1$, if
  \begin{align*}
  \text{(i)} & & \lim_{n \to \infty} \big( (A_n.F)(x) \big)^n \, = \; & G(x)
  \end{align*}
  holds, then necessarily
  \begin{align*}
    \lim_{n \to \infty} F (A_n^{-1}(x)) \; = \; 1 .
  \end{align*}
\end{lemma}
\begin{proof}
  % \uses{}
  % \leanok
  Otherwise $\big( F (A_n^{-1}(x)) \big)^n$ would have $0 \ne G(x)$
  as an accumulation point, contradicting the assumed limit (i).

  To wit, if for some $\delta > 0$ we would have
  $F (A_n^{-1}(x)) \le 1 - \delta$ for infinitely many $n$, then
  $0 \le \big( F (A_n^{-1}(x)) \big)^n \le (1 - \delta)^n $
  for those $n$, and since $(1 - \delta)^n \to 0$, we would
  get $\big( F (A_n^{-1}(x)) \big)^n \to 0 \ne G(x)$ along the
  subsequence of those $n$; a contradiction.
\end{proof}

\begin{lemma}[Taylor expansion limit modification]
  \label{lem:modify-limit-taylor}
  \lean{tendsto_smul_apply_smul_deriv_of_tendsto_atTop_of_tendsto_smul_apply_smul_deriv,
  tendsto_zero_of_tendsto_atTop_of_tendsto_smul}
  \leanok
  Let $S \subset \bR$ be a subset with $0 \in S$, and let $f_1, f_2 \colon S \to \bR$ be
  functions. Let also $(t_n)_{n \in \bN}$ be a sequence in $S$,
  and let $(m_n)_{n \in \bR}$ be a sequence of real numbers tending to infinity,
  $\lim_{n \to \infty} m_n = +\infty$.
  Assume that for $j = 1,2$
  \begin{itemize}
    \item $f_j(0)=0$;
    \item the derivative $f_j'(0)$ exists (derivative taken within the set $S$)
  \end{itemize}
  Assume further about $j = 1$ that
  \begin{itemize}
    \item $f_1'(0) \ne 0$;
    \item the limit $\lim_{n \to \infty} \big( m_n \, f_1 (t_n) \big)$ exists;
    \item for any $\delta > 0$ there exists an $\eps > 0$ such that
      if $|f_1(t)| < \eps$ (with $t \in S$) then $|t| < \delta$.
  \end{itemize}
  Denote $c = \frac{1}{f_1'(0)} \, \lim_{n \to \infty} \big( m_n \, f_1 (t_n) \big)$.
  Then we have $\lim_{n \to 0} t_n = 0$ and
  \begin{align*}
    \lim_{n \to \infty} \Big( m_n \, f_2 (t_n) \Big) \; = \; & c \, f_2'(0) .
  \end{align*}
\end{lemma}
\begin{proof}
  % \uses{}
  \leanok
  This is in principle straightforward:
  the assumptions are first checked to imply that $\lim_{n \to 0} t_n = 0$,
  and then one can just consider the first order Taylor expansions of the
  functions $f_j$ at $0$ given by the assumed existence of the derivatives
  (the key is $t_n \, = \, \frac{c}{m_n} + \oo \big(\frac{1}{m_n}\big)$).
\end{proof}

\begin{lemma}[Taylored version of the limit relation]
  \label{lem:taylored-ev-limit}
  \uses{def:cdf, lem:oriented-affine-action-on-cdf}
  \lean{log_ev_limit_iff_taylored_ev_limit}
  \leanok
  Let $F$ and $G$ be c.d.f.s, and $(A_n)_{n \in \N}$ a sequence of orientation
  preserving affine isomorphisms $A_n \in \OriAffR$. Then for any $x \in \bR$
  such that $0 < G(x) < 1$, the two conditions
  \begin{align*}
    \text{(ii)} & & \lim_{n \to \infty} n \, \log F (A_n^{-1}(x)) \, = \; & \log G(x) \\
    \text{(iii)} & & \lim_{n \to \infty} \Big( n \, \big(1 - F (A_n^{-1}(x)) \big) \Big)
      \, = \; & - \log G(x)
  \end{align*}
  are equivalent.
\end{lemma}
\begin{proof}
  \uses{lem:modify-limit-taylor}
  % \leanok

  Both implications {(ii) $\, \Rightarrow \,$ (iii)} and {(iii) $\, \Rightarrow \,$ (ii)}
  are proven similarly using Lemma~\ref{lem:modify-limit-taylor}.

  Assume (ii). Let $f_1(t) = - \log (1-t)$ and $f_2(t) = t$ and
  $S = [0,1) \subset \bR$, and $m_n = n$ and $t_n = 1 - F (A_n^{-1}(x))$.
  It is straightforward to check the assumptions of Lemma~\ref{lem:modify-limit-taylor},
  with
  \begin{align*}
    f_1'(0) \; = \; & \der{t} \Big|_{t=0} \Big( - \log (1 - t) \Big) \; = \; 1
  \end{align*}
  and $f_2'(0) = \mathrm{id}'(0) = 1$.
  The key assumption about the existence of
  the limit $\lim_{n \to \infty} \big( m_n \, f_1(t_n) \big)$ is given by (ii),
  and the conclusion is (iii).

  Similarly assuming (iii) we derive (ii) with Lemma~\ref{lem:modify-limit-taylor}
  just interchanging the roles of the two functions, i.e., now using
  $f_1(t) = t$ and $f_2(t) = - \log (1-t)$ instead.
\end{proof}

\begin{lemma}[Inverted Taylored version of the limit relation]
  \label{lem:inv-taylored-ev-limit}
  \uses{def:cdf, lem:oriented-affine-action-on-cdf}
  \lean{taylored_ev_limit_iff_oneDivOneSub_limit}
  \leanok
  Let $F$ and $G$ be c.d.f.s, and $(A_n)_{n \in \N}$ a sequence of orientation
  preserving affine isomorphisms $A_n \in \OriAffR$. Then for any $x \in \bR$
  such that $0 < G(x) < 1$, the two conditions
  \begin{align*}
    \text{(iii)} & & \lim_{n \to \infty} \Big( n \, \big(1 - F (A_n^{-1}(x)) \big) \Big)
      \, = \; & - \log G(x) \\
    \text{(iv)} & & \lim_{n \to \infty} \frac{1}{n \, \big(1 - F (A_n^{-1}(x)) \big)}
      \, = \; & \frac{1}{-\log G(x)}
  \end{align*}
  are equivalent.
\end{lemma}
\begin{proof}
  % \uses{}
  \leanok
  By assumption $G(x) \in (0,1)$ we have $-\log G(x) > 0$.

  The implication {(iii)~$\,\Rightarrow \,$~(iv)}
  can therefore be seen by applying $t \mapsto \frac{1}{t}$
  and using its continuity at $t = -\log G(x)$, and the
  converse implication {(iv)~$\,\Rightarrow \,$~(iii)} similarly by using continuity of
  $t \mapsto \frac{1}{t}$ at $t = \frac{1}{-\log G(x)}$.
\end{proof}

\begin{lemma}[Transformed version of the limit relation]
  \label{lem:transformed-ev-limit}
  \uses{def:one-div-one-sub-cdf, def:one-div-neg-log-cdf, lem:oriented-affine-action-on-cdf}
  \lean{oneDivSub_limit_iff}
  \leanok
  Let $F$ and $G$ be c.d.f.s, and $(A_n)_{n \in \N}$ a sequence of orientation
  preserving affine isomorphisms $A_n \in \OriAffR$. Then for any $x \in \bR$
  such that $0 < G(x) < 1$, the two conditions
  \begin{align*}
    \text{(iv)} & & \lim_{n \to \infty} \frac{1}{n \, \big(1 - F (A_n^{-1}(x)) \big)}
      \, = \; & \frac{1}{-\log G(x)} \\
    \text{(v)} & & \lim_{n \to \infty} \frac{1}{n} \oneDivOneSub{A_n . F} (x)
      \, = \; & \oneDivNegLog{G} (x)
  \end{align*}
  are equivalent.

  (See Definitions~\ref{def:one-div-one-sub-cdf}
  and \ref{def:one-div-neg-log-cdf} for the transforms involved in condition (v)).
\end{lemma}
\begin{proof}
  % \uses{}
  \leanok
  This is in principle straightforward, although certain cases need to be checked
  separately and the continuity of various natural extensions need to addressed.
\end{proof}

% def:one-div-neg-log-cdf


\begin{theorem}[Equivalent versions of the limit relation]
  \label{thm:tfae-ev-limit}
  \uses{def:one-div-one-sub-cdf, def:one-div-neg-log-cdf, lem:oriented-affine-action-on-cdf}
  \lean{tfae_ev_limit_relation}
  \leanok
  Let $F$ and $G$ be c.d.f.s, and $(A_n)_{n \in \N}$ a sequence of orientation
  preserving affine isomorphisms $A_n \in \OriAffR$. Then for any $x \in \bR$
  such that $0 < G(x) < 1$, the conditions
  \begin{align*}
    \text{(i)} & & \lim_{n \to \infty} \big( (A_n.F)(x) \big)^n \, = \; & G(x) \\
    \text{(ii)} & & \lim_{n \to \infty} n \, \log F (A_n^{-1}(x)) \, = \; & \log G(x) \\
    \text{(iii)} & & \lim_{n \to \infty} \Big( n \, \big(1 - F (A_n^{-1}(x)) \big) \Big)
      \, = \; & - \log G(x) \\
    \text{(iv)} & & \lim_{n \to \infty} \frac{1}{n \, \big(1 - F (A_n^{-1}(x)) \big)}
      \, = \; & \frac{1}{-\log G(x)} \\
    \text{(v)} & & \lim_{n \to \infty} \frac{1}{n} \oneDivOneSub{A_n . F} (x)
      \, = \; & \oneDivNegLog{G} (x)
  \end{align*}
  are equivalent.

  (See Definitions~\ref{def:one-div-one-sub-cdf}
  and \ref{def:one-div-neg-log-cdf} for the transforms involved in condition (v)).

  \emph{(More equivalent conditions are to be added; this is just a theorem to collect
  various equivalent phrasings.)}
\end{theorem}
\begin{proof}
  \uses{lem:taylored-ev-limit, lem:log-ev-limit, lem:inv-taylored-ev-limit,
  lem:transformed-ev-limit}
  \leanok
  This is just a combination of earlier results.
\end{proof}
