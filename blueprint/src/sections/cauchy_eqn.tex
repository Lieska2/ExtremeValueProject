\chapter{Cauchy-Hamel functional equation}

\section{Positive measure additive subgroups of the reals}

\begin{lemma}[Countably many connected components for an open set]
  \label{lem:countably-many-connected-components-of-open}
  %\uses{}
  \lean{IsOpen.countable_setOf_connectedComponentIn}
  \leanok
  Let $X$ be a locally connected separable space. Then any open subset $U \subseteq X$ has at
  most countably many connected components.
\end{lemma}
\begin{proof}
  % \uses{}
  \leanok
  (The proof is already formalized, see:
  \texttt{IsOpen.countable-setOf-connectedComponentIn}.)
\end{proof}

For subsets $A, B \subseteq \bR$, we use the following notation for pointwise
sum sets and difference sets:
\begin{align*}
  A + B & = \set{ a + b \; \big| \; a \in A, \, b \in B } , \\
  A - B & = \set{ a - b \; \big| \; a \in A, \, b \in B } .
\end{align*}

We denote by $\Leb$ the Lebesgue measure on $\bR$.

\begin{lemma}[Finding an interval with high overlap]
  \label{lem:exists-high-overlap-interval}
  %\uses{}
  \lean{exists_interval_measure_inter_gt_mul_measure}
  \leanok
  Let $A \subset \bR$ be a measurable set such that $0 < \Leb[A] < +\infty$.
  Then for any $r \in [0,1)$, there exists a nontrivial interval $J \subset \bR$
  (a subset of the real line which is connected and has nonempty interior)
  such that
  \begin{align*}
    \Leb[A \cap J] \; > \; r \, \Leb[J] .
  \end{align*}
\end{lemma}
\begin{proof}
  % \uses{}
  % \leanok
  Assume, without loss of generality, $0 < r < 1$.
  Since the Lebesgue measure is outer regular, we can find an open set $U \subset \bR$
  such that $A \subseteq U$ and $\Leb[U] < r^{-1} \, \Leb[A]$.

  The open set $U$ has at most countably many connected components
  (which are in fact open intervals); denote
  by $(U_i)_{i \in I}$ the indexed collection of them.

  Note that for at least one index $j \in I$ we
  have $\Leb[A \cap U_j] > r \Leb[U_j]$,
  because otherwise we get
  \begin{align*}
    \Leb[A] = \Leb[A \cap U]
    = \, & \sum_{i \in I} \Leb[A \cap U_i] \\
    \le \, & r \sum_{i \in I} \Leb[U_i] \\
    = \, & r \Leb[U],
  \end{align*}
  contradicting the choice of $U$.

  Now $J = U_j$ has the desired properties.
\end{proof}

\begin{lemma}[Shifts of a smaller interval contained in a larger interval]
  \label{lem:shift-interval-contained-in-larger}
  %\uses{}
  %\lean{}
  % \leanok % needs also `isPreconnected_of_Ioo_subset_of_subset_Icc` etc. before really done
  Let $I, J$ be nontrivial intervals, whose length satisfy
  \begin{align*}
    0 \; < \; \Leb[J] \; < \; \Leb[I] \; < \, +\infty .
  \end{align*}
  Then there exists an open interval $\Delta$ of length
  $\Leb[\Delta] = \Leb[I] - \Leb[J] > 0$ such that
  \begin{align*}
    t + J \; \subset \; I
    \qquad \text{ for any $t \in \Delta$.}
  \end{align*}
\end{lemma}
\begin{proof}
  % \uses{}
  % \leanok
  \ldots
\end{proof}

\begin{lemma}[Overlapping union of copies of an interval]
  \label{lem:overlapping-union-interval-copies}
  %\uses{}
  \lean{volume_union_add_self_ge_of_Ioo_subset, volume_union_add_self_le_of_subset_Icc,
    isConnected_of_Ioo_subset_of_subset_Icc}
  % \leanok % needs also `isPreconnected_of_Ioo_subset_of_subset_Icc` etc. before really done
  Let $J$ be a nontrivial interval of finite length ($0 \, < \, \Leb[J] \, < +\infty$).
  Let $c < 1$ and denote $\delta = c \Leb[J] \, < \, \Leb[J]$.
  Then for any $t \in (-\delta,\delta)$, the set
  $J' = (t+J) \cup J$ is an interval (connected set with nonempty interior)
  whose length satisfies the bound $\Leb[J'] \, < \, (1+c) \, \Leb[J]$.
\end{lemma}
\begin{proof}
  \uses{lem:shift-interval-contained-in-larger}
  % \leanok
  Denote $a = \inf J$ and $b = \sup J$. We have $-\infty < a < b < +\infty$ and
  \begin{align*}
    (a,b) \; \subseteq \; J \; \subseteq \; [a,b]
  \end{align*}
  and we have $\Leb[J] = b - a$.

  Let $t \in (+\delta,\delta)$, with $\delta = c \, \Leb[J] = c(b-a)$.
  Then we have the containments
  \begin{align*}
    \big( \min \{a,a+t\} , \max \{b,b+t\} \big)
      \; \subseteq \; (t+J) \cup J
      \; \subseteq \; \big[ \min \{a,a+t\} , \max \{b,b+t\} \big] .
  \end{align*}
  It in particular follows that $J' := (t+J) \cup J$ is an interval.

  If $t \ge 0$, then $J'$ is contained in
  $[a,b+t]$ which has length $b+t-a = (b-a) + t < \Leb[J] + c \, \Leb[J] = (1+c) \Leb[J]$.
  If $t < 0$, then $J'$ is contained in $[a+t,b]$ and one similarly gets a length bound.
\end{proof}


\begin{lemma}[Difference set of positive measure set contains an interval]
  \label{lem:difference-set-contains-interval}
  %\uses{}
  \lean{exists_Ioo_subset_diff_self_of_measure_pos}
  \leanok
  Let $A \subset \bR$ be a measurable set of positive Lebesgue measure.
  Then there exists a $\delta > 0$ such that
  \begin{align*}
    (-\delta,\delta) \, \subseteq \, A - A .
  \end{align*}
\end{lemma}
\begin{proof}
  \uses{lem:countably-many-connected-components-of-open, lem:exists-high-overlap-interval,
    lem:overlapping-union-interval-copies}
  % \leanok
  Pick a measurable subset $A_0 \subseteq A$ such that
  $0 < \Leb[A_0] < +\infty$.

  By Lemma~\ref{lem:exists-high-overlap-interval}, we can find a nontrivial interval~$J$
  such that $\Leb[A_0 \cap J] > \frac{3}{4} \Leb[J]$.
  Let $\delta = \frac{1}{2} \Leb[J]$.
  We claim that $(-\delta,\delta) \subseteq A_0 - A_0$ (which then
  clearly implies the assertion of the lemma).
  Indeed, suppose $t \in (-\delta,\delta)$.
  Then by Lemma~\ref{lem:overlapping-union-interval-copies},
  $(t+J) \cup J$ is an interval of length less than $\frac{3}{2} \Leb[J]$.
  Moreover, we have $A_0 \cap J \subseteq (t+J) \cup J$ and
  $t+(A_0 \cap J) \subseteq (t+J) \cup J$. Now note that
  \begin{align*}
    \Leb[t + (A_0 \cap J)] = \Leb[A_0 \cap J] > \frac{3}{4} \Leb[J] .
  \end{align*}
  If the sets $t + (A_0 \cap J)$ and $(A_0 \cap J)$ were disjoint, then
  the measure of $(A_0 \cap J) \cup (t+(A_0 \cap J))$ would thus be
  greater than $\frac{3}{2} \Leb[J]$, which is impossible given the length
  of the interval $(t+J) \cup J$ is less than $\frac{3}{2} \Leb[J]$.
  Therefore there exists a point $a \in (t + (A_0 \cap J)) \cap (A_0 \cap J)$.
  Denoting $a' = a - t$, we have $a, a' \in A_0 \cap J \subseteq A_0$
  and $t = a - a' \in A_0 - A_0 \subseteq A - A$. Since $t \in (-\delta,\delta)$
  was arbitrary, this proves the assertion.
\end{proof}

\begin{lemma}[Dividing a high overlap interval]
  \label{lem:dividing-high-operlap-interval}
  %\uses{}
  % \lean{}
  % \leanok
  Let $A$ be a measurable. Suppose that for some $a < b$,
  the interval $J = (a,b)$ satisfies
  $\Leb[A \cap J] \; > \;  r \, \Leb[J]$. %Denote $\delta = \Leb[J] = b-a$.
  Let $m \in \Npos$, and consider the subintervals
  \begin{align*}
    J_i = \Big( a + \frac{i}{(b-a)m} , \; a + \frac{i+1}{(b-a)m} \Big)
    \qquad \text{ for $i = 0, \ldots, m-1$.}
  \end{align*}
  Then for some $i$ we have $\Leb[A \cap J_i] \; > \;  r \, \Leb[J_i]$.
\end{lemma}
\begin{proof}
  % \uses{}
  % \leanok
\end{proof}

\begin{lemma}[Difference of two positive measure sets contains an interval]
  \label{lem:different-difference-set-contains-interval}
  % \uses{}
  % \lean{}
  % \leanok
  Let $A, B \subset \bR$ be two measurable sets of positive Lebesgue measure.
  Then the difference set $A - B$ contains a nontrivial open interval.
  %\begin{align*}
  %  (-\delta,\delta) \, \subseteq \, A - B .
  %\end{align*}
\end{lemma}
\begin{proof}
  \uses{lem:dividing-high-operlap-interval}
  % \leanok
  Pick measurable subsets $A_0 \subseteq A$ and $B_0 \subseteq B$ such that
  $0 < \Leb[A_0] < +\infty$ and $0 < \Leb[B_0] < +\infty$.

  By Lemma~\ref{lem:exists-high-overlap-interval}, we can find nontrivial intervals~$J, I$
  such that $\Leb[A_0 \cap J] > \frac{3}{4} \Leb[J]$
  and $\Leb[B_0 \cap I] > \frac{3}{4} \Leb[I]$.

  By Lemma~\ref{lem:dividing-high-operlap-interval}
  % we may divide
  % $I$ and $J$ to equal length subintervals, at least one of which inherits
  % this property, i.e.,
  for any $n,m \in \bN$ we can find intervals
  $I' \subset I$ and $J' \subset J$
  with $\Leb[I'] = \frac{1}{n} \Leb[I]$ and $\Leb[J'] = \frac{1}{m} \Leb[J]$
  and such that $\Leb[A_0 \cap J'] > \frac{3}{4} \Leb[J']$
  and $\Leb[B_0 \cap I'] > \frac{3}{4} \Leb[I']$.
  By choosing $n$ and $m$ suitably, we can ensure that
  \begin{align*}
    \frac{1}{2} \Leb[I'] \; \le \; \Leb[J'] \; < \; \Leb[I'] .
  \end{align*}

  Then by Lemma~\ref{lem:shift-interval-contained-in-larger}
  there exists an open interval $\Delta$ of length
  $\Leb[\Delta] \, > \, \Leb[I'] - \Leb[J']$ such that
  for all $t \in \Delta$ we have $t + J' \subset I'$.

  Consider a fixed $t \in \Delta$.
  Observe that $t + (B_0 \cap J) \subset I'$ and
  \begin{align*}
    \Leb[t + (B_0 \cap J')] \; = \; \Leb[B_0 \cap J']
      \; > \; \frac{3}{4} \Leb[J'] \; \ge \; \frac{3}{8} \Leb[I'] .
  \end{align*}
  We now claim that $A_0 \cap I'$ and $t + (B_0 \cap J')$ intersect.
  Their measures add up to at least
  \begin{align*}
    \Leb[A_0 \cap I'] + \Leb[t + (B_0 \cap J')]
    \; = \; & \Leb[A_0 \cap I'] + \Leb[B_0 \cap J'] \\
    \; > \; & \frac{3}{4} \Leb[I'] + \frac{3}{4} \Leb[J'] \\
    \; \ge \; & \frac{3}{4} \Leb[I'] + \frac{3}{8} \Leb[I']
    \; = \; & \frac{9}{8} \Leb[I'] .
  \end{align*}
  and both have been shown to be subsets of $I'$; therefore they
  cannot be disjoint.
  In particular there is a point $z \in (A_0 \cap I') \cap (t + B_0 \cap J')$,
  which means that
  \begin{align*}
    a = z = t + b
  \end{align*}
  for some $a \in A_0 \cap I'$ and $b \in B_0 \cap J'$. We solve
  $t = a - b \in A_0 - B_0 \subseteq A - B$. Since $t \in \Delta$
  was arbitrary, we have shown that the nontrivial open interval $\Delta$
  is contained in $A - B$.
\end{proof}
